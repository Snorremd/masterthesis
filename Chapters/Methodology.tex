%!TEX root = ../Thesis.tex
% Chapter Template

\chapter{Methodology} % Main chapter title

\label{Methodology} % Change X to a consecutive number; for referencing this chapter elsewhere, use \ref{ChapterX}

\lhead{Chapter \ref{Methodology}. \emph{Methodology}} % Change X to a consecutive number; this is for the header on each page - perhaps a shortened title

Research methodologies and standards used to test search and classification algorithms will make the foundation of this thesis' research methodology. While experiments in the information retrieval field not necessarily directly involves human subjects, there are still standards and methodologies in place to ensure that experimental results are valid. This chapter will describe the scientific approach used in the scientific field of information retrieval, and how this approach will be applied to this thesis work.

\section{Experimental Evaluation}
\label{ExperimentalEvaluation}
The configuration and parameter sets for the \CTC algorithm will be evaluated experimentally rather than analytically for reasons explained by \cite[p. 32]{Sebastiani2002}:
\begin{quote}The reason is that, in order to evaluate a system analytically (e.g., proving that the system is correct and complete), we would need a formal specification of the problem that the system is trying to solve (e.g., with respect to what correctness and completeness are defined), and the central notion of TC [Text Classification] (namely, that of membership of a document in a category) is, due to its subjective character, inherently nonformalizable. The experimental evaluation of a classifier usually measures its effectiveness (rather than its efficiency), that is, its ability to take the right classification decisions.
\end{quote}

Experimental evaluation have long been used and discussed in the field of information retrieval. One of the first experimental paradigms in information retrieval research, one that is still in use today, is the test collection evaluation paradigm introduced by The Cranfield research projects during the 60s \cite{Cleverdon1966}. The experimental methodology formed during these experiments are nicely summarized by \cite[p. 564]{Voorhees2005} who writes that:

\begin{quote}
At the core of this experimental methodology was the idea that live users could be removed from the evaluation loop, thus simplifying the evaluation and allowing researchers to run in vitro–style experiments in a laboratory with just their retrieval engine, a set of queries, a test collection, and a set of judgments (i.e., a list of relevant documents).
\end{quote}.

\cite[p. 33]{Cleverdon1966} give three requirements for using measurements of performance in experimental tests of information retrieval systems:
\begin{enumerate}
\item A document collection of known size to be used in the test;
\item A set of questions, together with decisions as to exactly which documents are relevant to each question;
\item A set of results of searches made in the test; these usually give the numbers of documents retrieved in the searches, divided into the relevant and non-relevant documents.
\end{enumerate}
These questions should be asked with regards to information retrieval experiments done on text search via queries, but can inspire similar questions for experiments done on text classification and clustering algorithms. Instead of forming questions and determining which documents are relevant to those question, one can form a set of categories and then decide which documents fall into which categories. Then clustering results can be divided into clusters, each cluster correct or incorrect.

\section{Corpora}
\label{Corpora}

When performing experimental research on information retrieval systems it is customary to use standard document collections or corpora as they are also known. There are several corpora available for text classification and clustering research. \cite{Baeza-Yates2011a} describe some of the corpora available for text classification research among them: Reuters-21578, RCV: Reuters Corpus Volumes, the OHSUMED reference collection and 20 NewsGroups. Some of them are briefly explained below.

Reuters, an international news agency have made several corpora that are available trough different sources. One Reuters corpus that have been much used in the text classification community is the \textit{Reuters 21578} corpus \cite{Lewis2004a}. The documents in this collection was collected from documents appearing on the Reuters newswire in 1987. The corpus was assembled and categorized by personnel from Reuters and Carnegie Group in 1987. This corpus thus resembles that used in the ``Recycling of news in the news papers 1999 - 2000'' research project.

Reuters have since made a new corpus that is likely to replace the Reuters 21578 corpus. This new corpus is divided into three volumes RCV1, RCV2 and TRC2. The first two volumes contain news stories from 96 - 97, and the last volume contains news stories from 08 to 09. RCV1 and TRC2 contain english news stories only, while RCV2 is multilingual \cite{NationalInstituteofStandardsandTechnology2004}. An article on use of the RCV1 corpus provide more details about how the data set can be used in evaluation text categorization systems. ``\textit{Reuters CorpusVolume I (RCV1) is an archive of over 800,000 manually categorized newswire stories recently made available by Reuters, Ltd. for research purposes. Use of this data for research on text categorization requires a detailed understanding of the real world constraints under which the data was produced.}'' \cite{Lewis2004}. 

This thesis will mainly focus on news corpora, but the LLI research group will also look at the blogosphere. With this in mind it would be interesting to use a standard blog collection for evaluation of the algorithm in future research. Two blog collections are used in the blog track in the TREC conference, the Blogs06 and Blogs08 collections.\begin{quote}
The TREC Blogs06 collection is a big sample of the blogosphere, and contains spam as well as possibly non-blogs, e.g. RSS feeds from news broadcasters. It was crawled over an eleven week period from 6th December 2005 until the 21st February 2006. The collection is 148GB in size [\dots] The collection was used in TREC 2006, 2007 and 2008 \cite{Macdonald2011}.
\end{quote} 

These corpora form a solid foundation for experimental evaluations and make it possible to replicate and compare results between research projects and researchers. But for this to be possible, it is necessary to use some formal evaluation measures that are employed by a majority of the research in the area of study. This will be the focus of the next section (Section~\ref{EvaluationMeasures}).

\section{Evaluation Measures}
\label{EvaluationMeasures}
There are some considerations to take when choosing evaluation measures. When performing experimental research it is important to use the same evaluation measures as those used in related research works to make the results comparable. In much of information retrieval research, text classification included, there are agreed upon measures that can be used while performing research. Such measures does not exist to the same extent for clustering algorithms. There seems to be a community of practice with regards to evaluation measures used for the \STC algorithm. An alternative evaluation measure used by the LLI group in their research will be compared to the other measures.

\subsection{Chim and Deng}

\cite{Chim2007} detail how they perform an experimental evaluation of their clustering result in some detail. \citeauthor{Chim2007} use the evaluation measures on the results from a hierarchical clustering algorithm. The results from hierarchical clustering algorithms and the \CTC algorithm are however similar in nature. and it will be possible to use measurement formulas described in their article when calculating the various performance measurements. These formulas have also been used by \cite{Rafi2011} when they compare the standard suffix tree clustering algorithm of \citeauthor{Oren1998} with the algorithm formulated by \citeauthor{Chim2008}. Their papers describe four standard measurements for clustering quality: precision, recall, F-Measure and overall F-Measure.

If you have the sets

\begin{displaymath}
C = \{C_{1}, C_{2}, \dots, C_{k}\}
\end{displaymath}
\begin{displaymath}
C^* = \{C_1^*, C_2^*, \dots, C_l^*\}
\end{displaymath}
\begin{displaymath}
D = \{D_{1}, D_{2}, \dots, D_{k}\}
\end{displaymath}

where \(C\) is the clusters produced by the algorithms on document set \(D\), and \(C^*\) is the ``correct'' classes of document set \(D\), then the recall, precision and F-measure of cluster \(j\) with respect to class \(i\) can be calculated as:

\begin{displaymath}
recall(i,j) = \frac{\vert C_{j} \cap C_i^* \vert}{\vert C_i^* \vert}
\end{displaymath}
\begin{displaymath}
precision(i,j) = \frac{\vert C_{j} \cap C_i^* \vert}{\vert C_{j} \vert}
\end{displaymath}
\begin{displaymath}
F-Measure(i,j) = \frac{2 \cdot precision(i,j) \cdot recall(i,j)}{precision(i,j) + recall(i,j)}
\end{displaymath}

\cite{Chim2007} do not provide any information as to whether the category set \(C^*\) is disjoint (i.e. whether one document can occur in several categories). The category set is therefore most likely not disjoint, or it depends on the collection. As was explained in the literature section recall aims to capture the fraction of positive results to the total number of correct results. In this context recall expresses the fraction of a category's documents the cluster contains. Precision shows the fraction of documents in a cluster that is correctly clustered given a category to the amount of documents in a cluster. Because precision and recall is not good measures by themselves (recall could be a 100\%, but often precision would then be very low) an F-Measure is often used in evaluation of text classifiers \cite{Baeza-Yates2011a}. The F-Measure is the harmonic mean between recall and precision and is high when both precision and recall is high \cite{Baeza-Yates2011b}. 

The precision, recall, and F-Measure measurements defined above are applied to one cluster and class at a time. In other words the F-Measure of a cluster j with regards to a class i might not be any good, but its F-Measure with regards to another class i' might be very good. \cite{Chim2007} define an overall F-Measure function that captures the F-Measures for all the correct classes defined for the document set. For this function only those clusters j which maximize the F-Measure score for class i are considered in the overall F-Measure score. The overall F-Measure is calculated using the function:

Given the formulas the overall F-Measure of the clusters \(C\) can be calculated using the formula:
\begin{displaymath}
F := \sum_{i=1}^{l} \frac{\vert C_i^* \vert}{\vert D \vert} \cdot \max_{j=1,\dots,k} \{F-Measure(i,j)\}
\end{displaymath}

\subsection{LII research group}
Write about Richards performance measures here


\subsection{Measurement comparison}
Compare the measurements here

\section{Experimental Research}
\label{ExperimentalResearch}
Rewrite and use corresponding section from project proposal. Try to flesh out the methodology a bit (how did I perform the testing, what where the hypotheses etc). Talk about experimental constraints and data used.