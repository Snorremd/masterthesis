%!TEX root = ../Thesis.tex

% Chapter Template
\chapter{Introduction} % Main chapter title

\label{Introduction}
%use \ref{Introduction}

\lhead{Chapter \ref{Introduction}. \emph{Introduction}} % Change X to a consecutive number; this is for the header on each page - perhaps a shortened title

%----------------------------------------------------------------------------------------
%	SECTION 1: Background
%----------------------------------------------------------------------------------------

% \section{Background} Should this be used as the main introduction before the motivation? ala Aleksander Larsen
Information Retrieval is a field in information, informatics, and computer science that revolves around retrieving and classifying information in order to make information more accessible. Theories and practices developed in this field drive many of the information retrieval systems we use in our every day lives such as Google Search, Duckduckgo, TinEye (an image search engine), file system search engines, and many more. One major area of information retrieval is classification and grouping (or clustering) of text documents or other information artefacts. Clustering will be the subject of this master thesis.

This master thesis started as a master thesis proposal by Richard Elling Moe. The proposal suggests that work be put into optimising the effectiveness (quality of results) of the \STC algorithm by means of adjusting its algorithm design and corresponding parameters. The \STC algorithm is presented by \textcite{Oren1998} in the paper \citetitle{Oren1998}. The algorithm is a clustering algorithm which extracts phrases from text documents and finds clusters by inserting these phrases into a suffix tree. \cite{Moe2014compact} have worked with the \STC algorithm in relation to a research project at the \deptname at the University of Bergen. The project is described in the paper \citetitle{Elgesem2009}, \cite{Elgesem2009}.

In this work they experimented with the \STC algorithm, and its parameters, and implemented a variation called \CTC with several modifications. The original \STC algorithm proposed by \textcite{Oren1998} is demonstrated in use on search engine results and showed good results on these kinds of data. In their work on a sample corpus (henceforth called ``Klimauken'' corpus), a corpus comprising articles from several big Norwegian newspapers, \cite{Moe2014compact} found that the \STC algorithm yielded poorer results. This can be explained as search engine results have been pre-filtered by the search engine and thus show some similarities from the get go. The news documents are quite varied in content and as such the default algorithm design and parameters of the \STC algorithm are not sufficient to get good results. This provides the background for this master thesis. In the thesis an approach to finding a more optimised algorithm design and better parameter values for the \CTC algorithm will be explored.

%----------------------------------------------------------------------------------------
%	SECTION 2: Motivation
%----------------------------------------------------------------------------------------

\section{Motivation}

The \STC algorithm has one important benefit over many of the traditional clustering algorithms. It is phrase based which means that it takes into account the position of words when comparing the similarity of documents. Many traditional clustering algorithms use the bag of words model where each document is considered a set of words and any positional properties are disregarded. The drawback is however that the \STC algorithm has shown poor performance on an unfiltered text corpus. It is of interest to see if the algorithm's performance can be improved by changing its algorithm design. There is evidence to suggest that an adaptation of the algorithm design and its parameters improve results. Manual tuning of the algorithm design has produced better results for a spesific corpus in existing research, \parencite{Moe2014,Moe2014compact}. This thesis will provide a way of improving the algorithm design of the \CTC algorithm for a given corpus through an automated optimisation algorithm. Such an automated optimisation algorithm could potentially make it possible for researchers to more easily adapt the \CTC algorithm to their corpora of choice.

% \subsection{Other motivations}
% I have previously taken two courses about information retrieval and web intelligence. This area of research is very interesting and this master thesis provided an opportunity to learn more about it. I also wanted to use previous knowledge in my master thesis, so I proposed to use the \GA to identify good parameter sets as genetic algorithms are well suited to the task of exploring large feature spaces. I also got the opportunity to learn a new programming language, Python, as this was already used by Richard Elling Moe.

%----------------------------------------------------------------------------------------
%	SECTION 3: Research Questions
%----------------------------------------------------------------------------------------

\section{Research question}

The main goal of this thesis is to demonstrate the feasibility of automatically optimising the algorithm design of the \CTC algorithm by means of a genetic algorithm.
Feasibility here encompasses two qualities, the effectiveness and the practicality of the algorithm. This gives way to two sub-goals:
\begin{enumerate}
\item The thesis will investigate whether the genetic algorithm is \emph{effective}, that is whether it improves the algorithm design and parameters of the \CTC algorithm for the ``Klimauken'' corpus so that it performs better than an average algorithm design.
\item It will also look into the \emph{practicality} of the algorithm, i.e. the feasibility of running the algorithm with regards to equipment, time, and knowledge.
\end{enumerate}

In addition to demonstrating the feasibility the thesis makes two important contributions:
\begin{inparaenum}[\itshape 1\upshape)]
\item It provides an optimised algorithm design and corresponding parameter values for the ``Klimauken'' corpus, and
\item it also suggest sensible parameter ranges for new corpora similar in nature to the ``Klimauken'' news corpus.
\end{inparaenum}

This thesis is for researchers and users of the \STC algorithm. With the optimisation algorithm discussed in this thesis they will be able to find algorithm designs optimised for their corpora. Demonstrating the feasibility of the optimisation algorithm is therefore important. An optimisation algorithm that requires too much processing power to work would be quite impractical to use. The thesis is also very much aimed at the research group working with the project discussed in \citetitle{Elgesem2009} and other information retrieval researchers.

This research is performed as part of a master thesis. As such certain temporal and economic constraints are imposed upon the scope and detail of the research. The tests were run on a single corpus, the ``Klimauken'' corpus. Additionally the tests were initially run on various personal machines which made time efficiency comparisons between algorithm designs difficult. The time element of optimisation is however an interesting one, and could be investigated further in future research.