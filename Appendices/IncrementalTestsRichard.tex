%!TEX root = ../Thesis.tex
% Incremental tests using Richard's parameters as base


\setstretch{1}
% Define bar chart colors
\definecolor{bblue}{HTML}{3366CC}
\definecolor{rred}{HTML}{DC3912}
\definecolor{oorange}{HTML}{FF9900}
\definecolor{ggreen}{HTML}{109618}
\definecolor{ppurple}{HTML}{990099}
\definecolor{tteal}{HTML}{0099C6}

% TREE TYPE TESTS
\begin{figure}[H]
  \begin{center}
\begin{tikzpicture}
  \begin{axis}[
    width  = 0.8*\textwidth,
    height = 4.55cm,
    % major x tick style = transparent,
    ybar=2*\pgflinewidth,
    bar width=6pt,
    ymajorgrids = true,
    ylabel = {Score},
    symbolic x coords={Suffix,Midslice,Rangeslice 0.1-1.0,5-slice},
    xtick = data,
    scaled y ticks = false,
    enlarge x limits=0.15,
    ymin=0,
    nodes near coords,
    nodes near coords align={horizontal},
    every node near coord/.append style={font=\tiny,rotate=90,color=black,anchor=west,/pgf/number format/fixed},
    enlarge y limits={upper,value=0.5},
    legend cell align=left,
    legend style={
      cells={anchor=east},
      legend pos=outer north east
    }
  ]

  \addplot [style={bblue,fill=bblue,mark=none}] table [col sep=semicolon,y=Precision] {Diagrams/Richard/testTreesRichard.csv};
  \addplot [style={rred,fill=rred,mark=none}] table [col sep=semicolon,y=Recall] {Diagrams/Richard/testTreesRichard.csv};
  \addplot [style={oorange,fill=oorange,mark=none}] table [col sep=semicolon,y=F-Measure] {Diagrams/Richard/testTreesRichard.csv};
  \addplot [style={ggreen,fill=ggreen,mark=none}] table [col sep=semicolon,y=Ground Truth 0] {Diagrams/Richard/testTreesRichard.csv};
  \addplot [style={ppurple,fill=ppurple,mark=none}] table [col sep=semicolon,y=Ground Truth Rep 0] {Diagrams/Richard/testTreesRichard.csv};
  \addplot [style={tteal,fill=tteal,mark=none}] table [col sep=semicolon,y=fMeasure 0] {Diagrams/Richard/testTreesRichard.csv};

  \legend{Precision,Recall,F-Measure,Ground truth 0,Ground truth rep 0, F-Measure 0}
  
  \end{axis}
\end{tikzpicture}
  \end{center}
  \caption{Performance of the \CTC algorithm for different expansion techniques.}
  \label{diag:treetypesrichard}
\end{figure}

% N-SLICE
\begin{figure}[H]
  \begin{center}
\begin{tikzpicture}
  \begin{axis}[
    width  = 0.8*\textwidth,
    height = 4.55cm,
    % major x tick style = transparent,
    xlabel = {N-Slice length},
    ylabel = {Score},
    %xtick = data,
    ymin=0,
    legend cell align=left,
    legend style={
      cells={anchor=east},
      legend pos=outer north east
    }
  ]

  \addplot [style={bblue,mark=none}] table [col sep=semicolon,y=Precision,x=n-slice] {Diagrams/Richard/testNSlices.csv};
  \addplot [style={rred,mark=none}] table [col sep=semicolon,y=Recall,x=n-slice] {Diagrams/Richard/testNSlices.csv};
  \addplot [style={oorange,mark=none}] table [col sep=semicolon,y=F-Measure,x=n-slice] {Diagrams/Richard/testNSlices.csv};
  \addplot [style={ggreen,mark=none}] table [col sep=semicolon,y=Ground Truth 0,x=n-slice] {Diagrams/Richard/testNSlices.csv};
  \addplot [style={ppurple,mark=none}] table [col sep=semicolon,y=Ground Truth Rep 0,x=n-slice] {Diagrams/Richard/testNSlices.csv};
  \addplot [style={tteal,mark=none}] table [col sep=semicolon,y=fMeasure 0,x=n-slice] {Diagrams/Richard/testNSlices.csv};

  \legend{Precision,Recall,F-Measure,Ground truth 0,Ground truth rep 0, F-Measure 0}
  
  \end{axis}
\end{tikzpicture}
  \end{center}
  \caption{Performance results of the \CTC algorithm for different lengths of n-slice expansion.}
  \label{diag:nslicesrichard}
\end{figure}

% RANGE SLICE TEST
\begin{figure}[H]
  \begin{center}
\begin{tikzpicture}
  \begin{axis}[
    width  = 0.8*\textwidth,
    height = 4.55cm,
    % major x tick style = transparent,
    ybar=2*\pgflinewidth,
    bar width=5pt,
    ymajorgrids = true,
    ylabel = {Scores},
    xlabel = {Range slice length},
    symbolic x coords={0.0-1.0,0.1-0.9,0.2-0.8,0.3-0.7,0.4-0.6,0.5-0.5},
    xtick = data,
    scaled y ticks = false,
    enlarge x limits=0.10,
    ymin=0,
    ymax=0.6,
    nodes near coords,
    nodes near coords align={horizontal},
    every node near coord/.append style={font=\tiny,rotate=90,color=black,anchor=west,/pgf/number format/fixed},
    enlarge y limits={upper,value=0.5},
    legend cell align=left,
    legend style={
      cells={anchor=east},
      legend pos=outer north east
    }
  ]

  \addplot [style={bblue,fill=bblue,mark=none}] table [col sep=semicolon,y=Precision] {Diagrams/Richard/testRangeSlices.csv};
  \addplot [style={rred,fill=rred,mark=none}] table [col sep=semicolon,y=Recall] {Diagrams/Richard/testRangeSlices.csv};
  \addplot [style={oorange,fill=oorange,mark=none}] table [col sep=semicolon,y=F-Measure] {Diagrams/Richard/testRangeSlices.csv};
  \addplot [style={ggreen,fill=ggreen,mark=none}] table [col sep=semicolon,y=Ground Truth 0] {Diagrams/Richard/testRangeSlices.csv};
  \addplot [style={ppurple,fill=ppurple,mark=none}] table [col sep=semicolon,y=Ground Truth Rep 0] {Diagrams/Richard/testRangeSlices.csv};
  \addplot [style={tteal,fill=tteal,mark=none}] table [col sep=semicolon,y=fMeasure 0] {Diagrams/Richard/testRangeSlices.csv};

  \legend{Precision,Recall,F-Measure,Ground truth 0,Ground truth rep 0, F-Measure 0}
  
  \end{axis}
\end{tikzpicture}
  \end{center}
  \caption{Performance of the \CTC algorithm for different ranges of range-slice expansion.}
  \label{diag:rangelicesrichard}
\end{figure}

% NUMBER OF TOP BASE CLUSTERS
\begin{figure}[H]
  \begin{center}
\begin{tikzpicture}
  \begin{axis}[
    width  = 0.8*\textwidth,
    height = 4.55cm,
    % major x tick style = transparent,
    xlabel = {Base cluster amount},
    ylabel = {Score},
    %xtick = data,
    ymin=0,
    xmin=0,
    legend cell align=left,
    legend style={
      cells={anchor=east},
      legend pos=outer north east
    }
  ]

  \addplot [style={bblue,mark=none}] table [col sep=semicolon,y=Precision,x=Basecluster-amount] {Diagrams/Richard/testBaseClusterAmounts.csv};
  \addplot [style={rred,mark=none}] table [col sep=semicolon,y=Recall,x=Basecluster-amount] {Diagrams/Richard/testBaseClusterAmounts.csv};
  \addplot [style={oorange,mark=none}] table [col sep=semicolon,y=F-Measure,x=Basecluster-amount] {Diagrams/Richard/testBaseClusterAmounts.csv};
  \addplot [style={ggreen,mark=none}] table [col sep=semicolon,y=Ground Truth 0,x=Basecluster-amount] {Diagrams/Richard/testBaseClusterAmounts.csv};
  \addplot [style={ppurple,mark=none}] table [col sep=semicolon,y=Ground Truth Rep 0,x=Basecluster-amount] {Diagrams/Richard/testBaseClusterAmounts.csv};
  \addplot [style={tteal,mark=none}] table [col sep=semicolon,y=fMeasure 0,x=Basecluster-amount] {Diagrams/Richard/testBaseClusterAmounts.csv};

  \legend{Precision,Recall,F-Measure,Ground truth 0,Ground truth rep 0, F-Measure 0}
  
  \end{axis}
\end{tikzpicture}
  \end{center}
  \caption{Performance of the \CTC algorithm fordifferent limits on top base clusters amount.}
  \label{diag:topbaseclustersrichard}
\end{figure}

% MIN TERM OCCURRENCE
\begin{figure}[H]
  \begin{center}
\begin{tikzpicture}
  \begin{axis}[
    width  = 0.8*\textwidth,
    height = 4.55cm,
    % major x tick style = transparent,
    xlabel = {Min term occurrence},
    ylabel = {Score},
    %xtick = data,
    ymin=0,
    xmin=0,
    xmax=200,
    legend cell align=left,
    legend style={
      cells={anchor=east},
      legend pos=outer north east
    }
  ]

  \addplot [style={bblue,mark=none}] table [col sep=semicolon,y=Precision,x=Min Term Occurrence] {Diagrams/Richard/testMinTermOccurrence.csv};
  \addplot [style={rred,mark=none}] table [col sep=semicolon,y=Recall,x=Min Term Occurrence] {Diagrams/Richard/testMinTermOccurrence.csv};
  \addplot [style={oorange,mark=none}] table [col sep=semicolon,y=F-Measure,x=Min Term Occurrence] {Diagrams/Richard/testMinTermOccurrence.csv};
  \addplot [style={ggreen,mark=none}] table [col sep=semicolon,y=Ground Truth 0,x=Min Term Occurrence] {Diagrams/Richard/testMinTermOccurrence.csv};
  \addplot [style={ppurple,mark=none}] table [col sep=semicolon,y=Ground Truth Rep 0,x=Min Term Occurrence] {Diagrams/Richard/testMinTermOccurrence.csv};
  \addplot [style={tteal,mark=none}] table [col sep=semicolon,y=fMeasure 0,x=Min Term Occurrence] {Diagrams/Richard/testMinTermOccurrence.csv};

  \legend{Precision,Recall,F-Measure,Ground truth 0,Ground truth rep 0, F-Measure 0}
  
  \end{axis}
\end{tikzpicture}
  \end{center}
  \caption{Performance of the \CTC algorithm for different limits on minimal term occurrence in collection.}
  \label{diag:mintermoccurrencesrichard}
\end{figure}

% MAX TERM RATIO
\begin{figure}[H]
  \begin{center}
\begin{tikzpicture}
  \begin{axis}[
    width  = 0.8*\textwidth,
    height = 4.55cm,
    % major x tick style = transparent,
    xlabel = {Max term ratio},
    ylabel = {Score},
    %xtick = data,
    ymin=0,
    xmin=0.1,
    xmax=1,
    legend cell align=left,
    legend style={
      cells={anchor=east},
      legend pos=outer north east
    }
  ]

  \addplot [style={bblue,mark=none}] table [col sep=semicolon,y=Precision,x=Max Term Ratio] {Diagrams/Richard/testMaxTermRatio.csv};
  \addplot [style={rred,mark=none}] table [col sep=semicolon,y=Recall,x=Max Term Ratio] {Diagrams/Richard/testMaxTermRatio.csv};
  \addplot [style={oorange,mark=none}] table [col sep=semicolon,y=F-Measure,x=Max Term Ratio] {Diagrams/Richard/testMaxTermRatio.csv};
  \addplot [style={ggreen,mark=none}] table [col sep=semicolon,y=Ground Truth 0,x=Max Term Ratio] {Diagrams/Richard/testMaxTermRatio.csv};
  \addplot [style={ppurple,mark=none}] table [col sep=semicolon,y=Ground Truth Rep 0,x=Max Term Ratio] {Diagrams/Richard/testMaxTermRatio.csv};
  \addplot [style={tteal,mark=none}] table [col sep=semicolon,y=fMeasure 0,x=Max Term Ratio] {Diagrams/Richard/testMaxTermRatio.csv};

  \legend{Precision,Recall,F-Measure,Ground truth 0,Ground truth rep 0, F-Measure 0}
  
  \end{axis}
\end{tikzpicture}
  \end{center}
  \caption{Performance of the \CTC algorithm for different limits on max term ratio in collection.}
  \label{diag:maxtermratiorichard}
\end{figure}

% Min Limit BC Score
\begin{figure}[H]
  \begin{center}
\begin{tikzpicture}
  \begin{axis}[
    width  = 0.8*\textwidth,
    height = 4.55cm,
    % major x tick style = transparent,
    xlabel = {Min Limit BC Score},
    ylabel = {Score},
    %xtick = data,
    ymin=0,
    xmin=0,
    xmax=15,
    legend cell align=left,
    legend style={
      cells={anchor=east},
      legend pos=outer north east
    }
  ]

  \addplot [style={bblue,mark=none}] table [col sep=semicolon,y=Precision,x=Min Limit] {Diagrams/Richard/testMinLimitBC.csv};
  \addplot [style={rred,mark=none}] table [col sep=semicolon,y=Recall,x=Min Limit] {Diagrams/Richard/testMinLimitBC.csv};
  \addplot [style={oorange,mark=none}] table [col sep=semicolon,y=F-Measure,x=Min Limit] {Diagrams/Richard/testMinLimitBC.csv};
  \addplot [style={ggreen,mark=none}] table [col sep=semicolon,y=Ground Truth 0,x=Min Limit] {Diagrams/Richard/testMinLimitBC.csv};
  \addplot [style={ppurple,mark=none}] table [col sep=semicolon,y=Ground Truth Rep 0,x=Min Limit] {Diagrams/Richard/testMinLimitBC.csv};
  \addplot [style={tteal,mark=none}] table [col sep=semicolon,y=fMeasure 0,x=Min Limit] {Diagrams/Richard/testMinLimitBC.csv};

  \legend{Precision,Recall,F-Measure,Ground truth 0,Ground truth rep 0, F-Measure 0}
  
  \end{axis}
\end{tikzpicture}
  \end{center}
  \caption{Performance of the \CTC algorithm for different min limit values for base cluster score with unbounded max limit (max limit = length of longest label).}
  \label{diag:minlimitbcscorerichard}
\end{figure}

% Max Limit BC Score
\begin{figure}[H]
  \begin{center}
\begin{tikzpicture}
  \begin{axis}[
    width  = 0.8*\textwidth,
    height = 4.55cm,
    % major x tick style = transparent,
    xlabel = {Max Limit BC Score},
    ylabel = {Score},
    %xtick = data,
    ymin=0,
    xmin=3,
    xmax=15,
    legend cell align=left,
    legend style={
      cells={anchor=east},
      legend pos=outer north east
    }
  ]

  \addplot [style={bblue,mark=none}] table [col sep=semicolon,y=Precision,x=Max Limit] {Diagrams/Richard/testMaxLimitBC.csv};
  \addplot [style={rred,mark=none}] table [col sep=semicolon,y=Recall,x=Max Limit] {Diagrams/Richard/testMaxLimitBC.csv};
  \addplot [style={oorange,mark=none}] table [col sep=semicolon,y=F-Measure,x=Max Limit] {Diagrams/Richard/testMaxLimitBC.csv};
  \addplot [style={ggreen,mark=none}] table [col sep=semicolon,y=Ground Truth 0,x=Max Limit] {Diagrams/Richard/testMaxLimitBC.csv};
  \addplot [style={ppurple,mark=none}] table [col sep=semicolon,y=Ground Truth Rep 0,x=Max Limit] {Diagrams/Richard/testMaxLimitBC.csv};
  \addplot [style={tteal,mark=none}] table [col sep=semicolon,y=fMeasure 0,x=Max Limit] {Diagrams/Richard/testMaxLimitBC.csv};

  \legend{Precision,Recall,F-Measure,Ground truth 0,Ground truth rep 0, F-Measure 0}
  
  \end{axis}
\end{tikzpicture}
  \end{center}
  \caption{Performance of the \CTC algorithm for different max limit values for base cluster score. Min limit set to \protect\citeauthor{Oren1998} default.}
  \label{diag:maxlimitbcscorerichard}
\end{figure}

% Drop singleton bc test
\begin{figure}[H]
  \begin{center}
\begin{tikzpicture}
  \begin{axis}[
    width  = 0.8*\textwidth,
    height = 4.55cm,
    % major x tick style = transparent,
    ybar=2*\pgflinewidth,
    bar width=8pt,
    ymajorgrids = true,
    ylabel = {Score},
    symbolic x coords={0,1},
    xtick = data,
    scaled y ticks = false,
    enlarge x limits=0.25,
    ymin=0,
    nodes near coords,
    nodes near coords align={horizontal},
    every node near coord/.append style={font=\tiny,rotate=90,color=black,anchor=west,/pgf/number format/fixed},
    enlarge y limits={upper,value=0.5},
    legend cell align=left,
    legend style={
      cells={anchor=east},
      legend pos=outer north east
    }
  ]

  \addplot [style={bblue,fill=bblue,mark=none}] table [col sep=semicolon,y=Precision] {Diagrams/Richard/testDropSingletonBC.csv};
  \addplot [style={rred,fill=rred,mark=none}] table [col sep=semicolon,y=Recall] {Diagrams/Richard/testDropSingletonBC.csv};
  \addplot [style={oorange,fill=oorange,mark=none}] table [col sep=semicolon,y=F-Measure] {Diagrams/Richard/testDropSingletonBC.csv};
  \addplot [style={ggreen,fill=ggreen,mark=none}] table [col sep=semicolon,y=Ground Truth 0] {Diagrams/Richard/testDropSingletonBC.csv};
  \addplot [style={ppurple,fill=ppurple,mark=none}] table [col sep=semicolon,y=Ground Truth Rep 0] {Diagrams/Richard/testDropSingletonBC.csv};
  \addplot [style={tteal,fill=tteal,mark=none}] table [col sep=semicolon,y=fMeasure 0] {Diagrams/Richard/testDropSingletonBC.csv};

  \legend{Precision,Recall,F-Measure,Ground truth 0,Ground truth rep 0, F-Measure 0}
  
  \end{axis}
\end{tikzpicture}
  \end{center}
  \caption{Performance of the \CTC algorithm for exclusion and inclusion of singleton base clusters.}
  \label{diag:dropsingletonbcrichard}
\end{figure}

% Drop one word clusters test
\begin{figure}[H]
  \begin{center}
\begin{tikzpicture}
  \begin{axis}[
    width  = 0.8*\textwidth,
    height = 4.55cm,
    % major x tick style = transparent,
    ybar=2*\pgflinewidth,
    bar width=8pt,
    ymajorgrids = true,
    ylabel = {Score},
    symbolic x coords={0,1},
    xtick = data,
    scaled y ticks = false,
    enlarge x limits=0.25,
    ymin=0,
    nodes near coords,
    nodes near coords align={horizontal},
    every node near coord/.append style={font=\tiny,rotate=90,color=black,anchor=west,/pgf/number format/fixed},
    enlarge y limits={upper,value=0.5},
    legend cell align=left,
    legend style={
      cells={anchor=east},
      legend pos=outer north east
    }
  ]

  \addplot [style={bblue,fill=bblue,mark=none}] table [col sep=semicolon,y=Precision] {Diagrams/Richard/testDropOneWordClusters.csv};
  \addplot [style={rred,fill=rred,mark=none}] table [col sep=semicolon,y=Recall] {Diagrams/Richard/testDropOneWordClusters.csv};
  \addplot [style={oorange,fill=oorange,mark=none}] table [col sep=semicolon,y=F-Measure] {Diagrams/Richard/testDropOneWordClusters.csv};
  \addplot [style={ggreen,fill=ggreen,mark=none}] table [col sep=semicolon,y=Ground Truth 0] {Diagrams/Richard/testDropOneWordClusters.csv};
  \addplot [style={ppurple,fill=ppurple,mark=none}] table [col sep=semicolon,y=Ground Truth Rep 0] {Diagrams/Richard/testDropOneWordClusters.csv};
  \addplot [style={tteal,fill=tteal,mark=none}] table [col sep=semicolon,y=fMeasure 0] {Diagrams/Richard/testDropOneWordClusters.csv};

  \legend{Precision,Recall,F-Measure,Ground truth 0,Ground truth rep 0, F-Measure 0}
  
  \end{axis}
\end{tikzpicture}
  \end{center}
  \caption{Performance of the \CTC algorithm on exclusion and inclusion of one word clusters.}
  \label{diag:droponewordclustersrichard}
\end{figure}

% Sort descending test
\begin{figure}[H]
  \begin{center}
\begin{tikzpicture}
  \begin{axis}[
    width  = 0.8*\textwidth,
    height = 4.55cm,
    % major x tick style = transparent,
    ybar=2*\pgflinewidth,
    bar width=8pt,
    ymajorgrids = true,
    ylabel = {Score},
    symbolic x coords={0,1},
    xtick = data,
    scaled y ticks = false,
    enlarge x limits=0.25,
    ymin=0,
    nodes near coords,
    nodes near coords align={horizontal},
    every node near coord/.append style={font=\tiny,rotate=90,color=black,anchor=west,/pgf/number format/fixed},
    enlarge y limits={upper,value=0.5},
    legend cell align=left,
    legend style={
      cells={anchor=east},
      legend pos=outer north east
    }
  ]

  \addplot [style={bblue,fill=bblue,mark=none}] table [col sep=semicolon,y=Precision] {Diagrams/Richard/testSortDescending.csv};
  \addplot [style={rred,fill=rred,mark=none}] table [col sep=semicolon,y=Recall] {Diagrams/Richard/testSortDescending.csv};
  \addplot [style={oorange,fill=oorange,mark=none}] table [col sep=semicolon,y=F-Measure] {Diagrams/Richard/testSortDescending.csv};
  \addplot [style={ggreen,fill=ggreen,mark=none}] table [col sep=semicolon,y=Ground Truth 0] {Diagrams/Richard/testSortDescending.csv};
  \addplot [style={ppurple,fill=ppurple,mark=none}] table [col sep=semicolon,y=Ground Truth Rep 0] {Diagrams/Richard/testSortDescending.csv};
  \addplot [style={tteal,fill=tteal,mark=none}] table [col sep=semicolon,y=fMeasure 0] {Diagrams/Richard/testSortDescending.csv};

  \legend{Precision,Recall,F-Measure,Ground truth 0,Ground truth rep 0, F-Measure 0}
  
  \end{axis}
\end{tikzpicture}
  \end{center}
  \caption{Performance of the \CTC algorithm when base clusters are sorted in descending and acending order.}
  \label{diag:sortdescendingrichard}
\end{figure}

% Text amount
\begin{figure}[H]
  \begin{center}
\begin{tikzpicture}
  \begin{axis}[
    width  = 0.8*\textwidth,
    height = 4.55cm,
    % major x tick style = transparent,
    xlabel = {Article Text Amount},
    xmin=0,
    xmax=1,
    ylabel = {Score},
    %xtick = data,
    ymin=0,
    legend cell align=left,
    legend style={
      cells={anchor=east},
      legend pos=outer north east
    }
  ]

  \addplot [style={bblue,mark=none}] table [col sep=semicolon,y=Precision,x=Article Text Amount] {Diagrams/Richard/testArticleTextAmount.csv};
  \addplot [style={rred,mark=none}] table [col sep=semicolon,y=Recall,x=Article Text Amount] {Diagrams/Richard/testArticleTextAmount.csv};
  \addplot [style={oorange,mark=none}] table [col sep=semicolon,y=F-Measure,x=Article Text Amount] {Diagrams/Richard/testArticleTextAmount.csv};
  \addplot [style={ggreen,mark=none}] table [col sep=semicolon,y=Ground Truth 0,x=Article Text Amount] {Diagrams/Richard/testArticleTextAmount.csv};
  \addplot [style={ppurple,mark=none}] table [col sep=semicolon,y=Ground Truth Rep 0,x=Article Text Amount] {Diagrams/Richard/testArticleTextAmount.csv};
  \addplot [style={tteal,mark=none}] table [col sep=semicolon,y=fMeasure 0,x=Article Text Amount] {Diagrams/Richard/testArticleTextAmount.csv};

  \legend{Precision,Recall,F-Measure,Ground truth 0,Ground truth rep 0, F-Measure 0}
  
  \end{axis}
\end{tikzpicture}
  \end{center}
  \caption{Performance of the \CTC algorithm for different amounts of article text.}
  \label{diag:textamountrichard}
\end{figure}

% TEXT TYPE TESTS
\begin{figure}[H]
  \begin{center}
\begin{tikzpicture}
  \begin{axis}[
    width  = 0.8*\textwidth,
    height = 4.55cm,
    % major x tick style = transparent,
    ybar=2*\pgflinewidth,
    bar width=6pt,
    ymajorgrids = true,
    ylabel = {Score},
    xlabel = {Text types included},
    symbolic x coords={All,Frontpage,Article sans bread text,Article with bread text,Article text},
    x tick label style={font=\small,text width=1.7cm,align=center},
    xtick = data,
    scaled y ticks = false,
    enlarge x limits=0.10,
    ymin=0,
    nodes near coords,
    nodes near coords align={horizontal},
    every node near coord/.append style={font=\tiny,rotate=90,color=black,anchor=west,/pgf/number format/fixed},
    enlarge y limits={upper,value=0.5},
    legend cell align=left,
    legend style={
      cells={anchor=east},
      legend pos=outer north east
    }
  ]

  \addplot [style={bblue,fill=bblue,mark=none}] table [col sep=semicolon,y=Precision] {Diagrams/Richard/testTextTypes.csv};
  \addplot [style={rred,fill=rred,mark=none}] table [col sep=semicolon,y=Recall] {Diagrams/Richard/testTextTypes.csv};
  \addplot [style={oorange,fill=oorange,mark=none}] table [col sep=semicolon,y=F-Measure] {Diagrams/Richard/testTextTypes.csv};
  \addplot [style={ggreen,fill=ggreen,mark=none}] table [col sep=semicolon,y=Ground Truth 0] {Diagrams/Richard/testTextTypes.csv};
  \addplot [style={ppurple,fill=ppurple,mark=none}] table [col sep=semicolon,y=Ground Truth Rep 0] {Diagrams/Richard/testTextTypes.csv};
  \addplot [style={tteal,fill=tteal,mark=none}] table [col sep=semicolon,y=fMeasure 0] {Diagrams/Richard/testTextTypes.csv};

  \legend{Precision,Recall,F-Measure,Ground truth 0,Ground truth rep 0, F-Measure 0}
  
  \end{axis}
\end{tikzpicture}
  \end{center}
  \caption{Performance of the \CTC algorithm for inclusion of different types of texts.}
  \label{diag:texttypesrichard}
\end{figure}

% SIMILARITY METHODS TESTS
\begin{figure}[H]
  \begin{center}
\begin{tikzpicture}
  \begin{axis}[
    width  = 0.8*\textwidth,
    height = 4.55cm,
    % major x tick style = transparent,
    ybar=2*\pgflinewidth,
    bar width=8pt,
    ymajorgrids = true,
    ylabel = {Score},
    xlabel = {Similarity methods},
    symbolic x coords={Jaccard,Cosine,Amendment1C},
    xtick = data,
    scaled y ticks = false,
    enlarge x limits=0.20,
    ymin=0,
    nodes near coords,
    nodes near coords align={horizontal},
    every node near coord/.append style={font=\tiny,rotate=90,color=black,anchor=west, /pgf/number format/fixed},
    enlarge y limits={upper,value=0.5},
    legend cell align=left,
    legend style={
      cells={anchor=east},
      legend pos=outer north east
    }
  ]

  \addplot [style={bblue,fill=bblue,mark=none}] table [col sep=semicolon,y=Precision] {Diagrams/Richard/testSimilarityMethods.csv};
  \addplot [style={rred,fill=rred,mark=none}] table [col sep=semicolon,y=Recall] {Diagrams/Richard/testSimilarityMethods.csv};
  \addplot [style={oorange,fill=oorange,mark=none}] table [col sep=semicolon,y=F-Measure] {Diagrams/Richard/testSimilarityMethods.csv};
  \addplot [style={ggreen,fill=ggreen,mark=none}] table [col sep=semicolon,y=Ground Truth 0] {Diagrams/Richard/testSimilarityMethods.csv};
  \addplot [style={ppurple,fill=ppurple,mark=none}] table [col sep=semicolon,y=Ground Truth Rep 0] {Diagrams/Richard/testSimilarityMethods.csv};
  \addplot [style={tteal,fill=tteal,mark=none}] table [col sep=semicolon,y=fMeasure 0] {Diagrams/Richard/testSimilarityMethods.csv};

  \legend{Precision,Recall,F-Measure,Ground truth 0,Ground truth rep 0, F-Measure 0}
  
  \end{axis}
\end{tikzpicture}
  \end{center}
  \caption{Performance of the \CTC algorithm for different similarity methods.}
  \label{diag:similaritymethodsrichard}
\end{figure}

% JACCARD THRESHOLD
\begin{figure}[H]
  \begin{center}
\begin{tikzpicture}
  \begin{axis}[
    % Sizing
    width  = 0.8*\textwidth,
    height = 4.55cm,
    % Data
    xlabel = {Jaccard Coefficient Threshold},
    xmin=0,
    xmax=1,
    ymin=0,
    % Labeling
    ylabel = {Score},
    legend cell align=left,
    legend style={
      cells={anchor=east},
      legend pos=outer north east
    }
  ]

  \addplot [style={bblue,mark=none}] table [col sep=semicolon,y=Precision,x=Threshold] {Diagrams/Richard/testJaccardSimilarity.csv};
  \addplot [style={rred,mark=none}] table [col sep=semicolon,y=Recall,x=Threshold] {Diagrams/Richard/testJaccardSimilarity.csv};
  \addplot [style={oorange,mark=none}] table [col sep=semicolon,y=F-Measure,x=Threshold] {Diagrams/Richard/testJaccardSimilarity.csv};
  \addplot [style={ggreen,mark=none}] table [col sep=semicolon,y=Ground Truth 0,x=Threshold] {Diagrams/Richard/testJaccardSimilarity.csv};
  \addplot [style={ppurple,mark=none}] table [col sep=semicolon,y=Ground Truth Rep 0,x=Threshold] {Diagrams/Richard/testJaccardSimilarity.csv};
  \addplot [style={tteal,mark=none}] table [col sep=semicolon,y=fMeasure 0,x=Threshold] {Diagrams/Richard/testJaccardSimilarity.csv};

  \legend{Precision,Recall,F-Measure,Ground truth 0,Ground truth rep 0, F-Measure 0}
  
  \end{axis}
\end{tikzpicture}
  \end{center}
  \caption{Performance of the \CTC algorithm for different Jaccard Coefficient thresholds.}
  \label{diag:jaccardthresholdrichard}
\end{figure}

% COSINE THRESHOLD
\begin{figure}[H]
  \begin{center}
\begin{tikzpicture}
  \begin{axis}[
    % Sizing
    width  = 0.8*\textwidth,
    height = 4.55cm,
    % Data
    xlabel = {Cosine Threshold},
    xmin=0,
    xmax=1,
    ymin=0,
    % Labeling
    ylabel = {Score},
    legend cell align=left,
    legend style={
      cells={anchor=east},
      legend pos=outer north east
    }
  ]

  \addplot [style={bblue,mark=none}] table [col sep=semicolon,y=Precision,x=Cosine Threshold] {Diagrams/Richard/testCosineSimilarity.csv};
  \addplot [style={rred,mark=none}] table [col sep=semicolon,y=Recall,x=Cosine Threshold] {Diagrams/Richard/testCosineSimilarity.csv};
  \addplot [style={oorange,mark=none}] table [col sep=semicolon,y=F-Measure,x=Cosine Threshold] {Diagrams/Richard/testCosineSimilarity.csv};
  \addplot [style={ggreen,mark=none}] table [col sep=semicolon,y=Ground Truth 0,x=Cosine Threshold] {Diagrams/Richard/testCosineSimilarity.csv};
  \addplot [style={ppurple,mark=none}] table [col sep=semicolon,y=Ground Truth Rep 0,x=Cosine Threshold] {Diagrams/Richard/testCosineSimilarity.csv};
  \addplot [style={tteal,mark=none}] table [col sep=semicolon,y=fMeasure 0,x=Cosine Threshold] {Diagrams/Richard/testCosineSimilarity.csv};

  \legend{Precision,Recall,F-Measure,Ground truth 0,Ground truth rep 0, F-Measure 0}
  
  \end{axis}
\end{tikzpicture}
  \end{center}
  \caption{Performance of the \CTC algorithm for different Cosine Similarity thresholds.}
  \label{diag:cosinethresholdrichard}
\end{figure}

% Average corpus frequency limit
\begin{figure}[H]
  \begin{center}
\begin{tikzpicture}
  \begin{axis}[
    width  = 0.8*\textwidth,
    height = 4.55cm,
    % major x tick style = transparent,
    xlabel = {Avg corpus frequency limit},
    ylabel = {Score},
    %xtick = data,
    ymin=0,
    xmin=0,
    xmax=500,
    legend cell align=left,
    legend style={
      cells={anchor=east},
      legend pos=outer north east
    }
  ]

  \addplot [style={bblue,mark=none}] table [col sep=semicolon,y=Precision,x=Avg CF limit] {Diagrams/Richard/testAmendment1CSimilarityAvgCF.csv};
  \addplot [style={rred,mark=none}] table [col sep=semicolon,y=Recall,x=Avg CF limit] {Diagrams/Richard/testAmendment1CSimilarityAvgCF.csv};
  \addplot [style={oorange,mark=none}] table [col sep=semicolon,y=F-Measure,x=Avg CF limit] {Diagrams/Richard/testAmendment1CSimilarityAvgCF.csv};
  \addplot [style={ggreen,mark=none}] table [col sep=semicolon,y=Ground Truth 0,x=Avg CF limit] {Diagrams/Richard/testAmendment1CSimilarityAvgCF.csv};
  \addplot [style={ppurple,mark=none}] table [col sep=semicolon,y=Ground Truth Rep 0,x=Avg CF limit] {Diagrams/Richard/testAmendment1CSimilarityAvgCF.csv};
  \addplot [style={tteal,mark=none}] table [col sep=semicolon,y=fMeasure 0,x=Avg CF limit] {Diagrams/Richard/testAmendment1CSimilarityAvgCF.csv};

  \legend{Precision,Recall,F-Measure,Ground truth 0,Ground truth rep 0, F-Measure 0}
  
  \end{axis}
\end{tikzpicture}
  \end{center}
  \caption{Performance of the \CTC algorithm for different limits on max average corpus frequency in Amendment1C.}
  \label{diag:avgcfamendment1richard}
\end{figure}

% Base cluster intersect min limit
\begin{figure}[H]
  \begin{center}
\begin{tikzpicture}
  \begin{axis}[
    width  = 0.8*\textwidth,
    height = 4.55cm,
    % major x tick style = transparent,
    xlabel = {Min label intersect limit},
    ylabel = {Score},
    %xtick = data,
    ymin=0,
    xmin=0,
    xmax=50,
    legend cell align=left,
    legend style={
      cells={anchor=east},
      legend pos=outer north east
    }
  ]

  \addplot [style={bblue,mark=none}] table [col sep=semicolon,y=Precision,x=Min intersect limit] {Diagrams/Richard/testAmendment1CSimilarityIntersect.csv};
  \addplot [style={rred,mark=none}] table [col sep=semicolon,y=Recall,x=Min intersect limit] {Diagrams/Richard/testAmendment1CSimilarityIntersect.csv};
  \addplot [style={oorange,mark=none}] table [col sep=semicolon,y=F-Measure,x=Min intersect limit] {Diagrams/Richard/testAmendment1CSimilarityIntersect.csv};
  \addplot [style={ggreen,mark=none}] table [col sep=semicolon,y=Ground Truth 0,x=Min intersect limit] {Diagrams/Richard/testAmendment1CSimilarityIntersect.csv};
  \addplot [style={ppurple,mark=none}] table [col sep=semicolon,y=Ground Truth Rep 0,x=Min intersect limit] {Diagrams/Richard/testAmendment1CSimilarityIntersect.csv};
  \addplot [style={tteal,mark=none}] table [col sep=semicolon,y=fMeasure 0,x=Min intersect limit] {Diagrams/Richard/testAmendment1CSimilarityIntersect.csv};

  \legend{Precision,Recall,F-Measure,Ground truth 0,Ground truth rep 0, F-Measure 0}
  
  \end{axis}
\end{tikzpicture}
  \end{center}
  \caption{Performance of the \CTC algorithm for different minimum limits on base cluster label intersect in Amendment1C.}
  \label{diag:minintersectamendment1crichard}
\end{figure}