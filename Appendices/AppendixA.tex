%!TEX root = ../Thesis.tex
% Appendix A

\chapter{Incremental test results} % Main appendix title

\label{AppendixA} % For referencing this appendix elsewhere, use \ref{AppendixA}

\lhead{Appendix A. \emph{Incremental Test Results}} % This is for the header on each page - perhaps a shortened title

This appendix contains diagrams of all incremental test results. It serves as a reference for discussion and descriptions in section~\ref{Testing}. The appendix is divided into two main sections. The first contains results from incremental tests using the parameters specified by \cite{Oren1998} as base values. The second section contains results using the parameters suggested by \supervisor as base values.

\section{Oren \& Etzioni parameters}

% Define bar chart colors
%
\definecolor{bblue}{HTML}{3366CC}
\definecolor{rred}{HTML}{DC3912}
\definecolor{oorange}{HTML}{FF9900}
\definecolor{ggreen}{HTML}{109618}
\definecolor{ppurple}{HTML}{990099}
\definecolor{tteal}{HTML}{0099C6}

% TREE TYPE TESTS
\begin{figure}[H]
  \begin{center}
\begin{tikzpicture}
  \begin{axis}[
    width  = 0.8*\textwidth,
    height = 4.5cm,
    % major x tick style = transparent,
    ybar=2*\pgflinewidth,
    bar width=8pt,
    ymajorgrids = true,
    ylabel = {Score},
    symbolic x coords={Suffix,Midslice,Rangeslice 0.1-1.0,5-slice},
    xtick = data,
    scaled y ticks = false,
    enlarge x limits=0.25,
    ymin=0,
    legend cell align=left,
    legend style={
      cells={anchor=east},
      legend pos=outer north east
    }
  ]

  \addplot [style={bblue,fill=bblue,mark=none}] table [col sep=semicolon,y=Precision] {Diagrams/testTrees.csv};
  \addplot [style={rred,fill=rred,mark=none}] table [col sep=semicolon,y=Recall] {Diagrams/testTrees.csv};
  \addplot [style={oorange,fill=oorange,mark=none}] table [col sep=semicolon,y=F-Measure] {Diagrams/testTrees.csv};
  \addplot [style={ggreen,fill=ggreen,mark=none}] table [col sep=semicolon,y=Ground Truth 0] {Diagrams/testTrees.csv};
  \addplot [style={ppurple,fill=ppurple,mark=none}] table [col sep=semicolon,y=Ground Truth Rep 0] {Diagrams/testTrees.csv};
  \addplot [style={tteal,fill=tteal,mark=none}] table [col sep=semicolon,y=fMeasure 0] {Diagrams/testTrees.csv};

  \legend{Precision,Recall,F-Measure,Ground truth 0,Ground truth rep 0, F-Measure 0}
  
  \end{axis}
\end{tikzpicture}
  \end{center}
  \caption{Incremental results for different expansion techniques}
  \label{diag:treetypesetzioni}
\end{figure}

% N-SLICE
\begin{figure}[H]
  \begin{center}
\begin{tikzpicture}
  \begin{axis}[
    width  = 0.8*\textwidth,
    height = 4.5cm,
    % major x tick style = transparent,
    xlabel = {N-Slice length},
    ylabel = {Score},
    %xtick = data,
    ymin=0,
    legend cell align=left,
    legend style={
      cells={anchor=east},
      legend pos=outer north east
    }
  ]

  \addplot [style={bblue,mark=none}] table [col sep=semicolon,y=Precision,x=n-slice] {Diagrams/testNSlices.csv};
  \addplot [style={rred,mark=none}] table [col sep=semicolon,y=Recall,x=n-slice] {Diagrams/testNSlices.csv};
  \addplot [style={oorange,mark=none}] table [col sep=semicolon,y=F-Measure,x=n-slice] {Diagrams/testNSlices.csv};
  \addplot [style={ggreen,mark=none}] table [col sep=semicolon,y=Ground Truth 0,x=n-slice] {Diagrams/testNSlices.csv};
  \addplot [style={ppurple,mark=none}] table [col sep=semicolon,y=Ground Truth Rep 0,x=n-slice] {Diagrams/testNSlices.csv};
  \addplot [style={tteal,mark=none}] table [col sep=semicolon,y=fMeasure 0,x=n-slice] {Diagrams/testNSlices.csv};

  \legend{Precision,Recall,F-Measure,Ground truth 0,Ground truth rep 0, F-Measure 0}
  
  \end{axis}
\end{tikzpicture}
  \end{center}
  \caption{Incremental results for different n values of n-slice expansion}
  \label{diag:nslicesetzioni}
\end{figure}

% RANGE SLICE TEST
\begin{figure}[H]
  \begin{center}
\begin{tikzpicture}
  \begin{axis}[
     width  = 0.8*\textwidth,
    height = 4.5cm,
    % major x tick style = transparent,
    ybar=2*\pgflinewidth,
    bar width=5pt,
    ymajorgrids = true,
    ylabel = {Score},
    symbolic x coords={0.0-1.0,0.1-0.9,0.2-0.8,0.3-0.7,0.4-0.6,0.5-0.5},
    xtick = data,
    scaled y ticks = false,
    enlarge x limits=0.10,
    ymin=0,
    ymax=0.25,
    legend cell align=left,
    legend style={
      cells={anchor=east},
      legend pos=outer north east
    }
  ]

  \addplot [style={bblue,fill=bblue,mark=none}] table [col sep=semicolon,y=Precision] {Diagrams/testRangeSlices.csv};
  \addplot [style={rred,fill=rred,mark=none}] table [col sep=semicolon,y=Recall] {Diagrams/testRangeSlices.csv};
  \addplot [style={oorange,fill=oorange,mark=none}] table [col sep=semicolon,y=F-Measure] {Diagrams/testRangeSlices.csv};
  \addplot [style={ggreen,fill=ggreen,mark=none}] table [col sep=semicolon,y=Ground Truth 0] {Diagrams/testRangeSlices.csv};
  \addplot [style={ppurple,fill=ppurple,mark=none}] table [col sep=semicolon,y=Ground Truth Rep 0] {Diagrams/testRangeSlices.csv};
  \addplot [style={tteal,fill=tteal,mark=none}] table [col sep=semicolon,y=fMeasure 0] {Diagrams/testRangeSlices.csv};

  \legend{Precision,Recall,F-Measure,Ground truth 0,Ground truth rep 0, F-Measure 0}
  
  \end{axis}
\end{tikzpicture}
  \end{center}
  \caption{Incremental results for different ranges of range-slice expansion}
  \label{diag:rangeliceetzioni}
\end{figure}

% NUMBER OF TOP BASE CLUSTERS
\begin{figure}[H]
  \begin{center}
\begin{tikzpicture}
  \begin{axis}[
    width  = 0.8*\textwidth,
    height = 4.5cm,
    % major x tick style = transparent,
    xlabel = {Base cluster amount},
    ylabel = {Score},
    %xtick = data,
    ymin=0,
    legend cell align=left,
    legend style={
      cells={anchor=east},
      legend pos=outer north east
    }
  ]

  \addplot [style={bblue,mark=none}] table [col sep=semicolon,y=Precision,x=Basecluster-amount] {Diagrams/testBaseClusterAmounts.csv};
  \addplot [style={rred,mark=none}] table [col sep=semicolon,y=Recall,x=Basecluster-amount] {Diagrams/testBaseClusterAmounts.csv};
  \addplot [style={oorange,mark=none}] table [col sep=semicolon,y=F-Measure,x=Basecluster-amount] {Diagrams/testBaseClusterAmounts.csv};
  \addplot [style={ggreen,mark=none}] table [col sep=semicolon,y=Ground Truth 0,x=Basecluster-amount] {Diagrams/testBaseClusterAmounts.csv};
  \addplot [style={ppurple,mark=none}] table [col sep=semicolon,y=Ground Truth Rep 0,x=Basecluster-amount] {Diagrams/testBaseClusterAmounts.csv};
  \addplot [style={tteal,mark=none}] table [col sep=semicolon,y=fMeasure 0,x=Basecluster-amount] {Diagrams/testBaseClusterAmounts.csv};

  \legend{Precision,Recall,F-Measure,Ground truth 0,Ground truth rep 0, F-Measure 0}
  
  \end{axis}
\end{tikzpicture}
  \end{center}
  \caption{Incremental results for different limits on top base clusters amount}
  \label{diag:topbaseclustersetzioni}
\end{figure}

% MIN TERM OCCURRENCE
\begin{figure}[H]
  \begin{center}
\begin{tikzpicture}
  \begin{axis}[
    width  = 0.8*\textwidth,
    height = 4.5cm,
    % major x tick style = transparent,
    xlabel = {Min term occurrence},
    ylabel = {Score},
    %xtick = data,
    ymin=0,
    legend cell align=left,
    legend style={
      cells={anchor=east},
      legend pos=outer north east
    }
  ]

  \addplot [style={bblue,mark=none}] table [col sep=semicolon,y=Precision,x=Min Term Occurrence] {Diagrams/testMinTermOccurrence.csv};
  \addplot [style={rred,mark=none}] table [col sep=semicolon,y=Recall,x=Min Term Occurrence] {Diagrams/testMinTermOccurrence.csv};
  \addplot [style={oorange,mark=none}] table [col sep=semicolon,y=F-Measure,x=Min Term Occurrence] {Diagrams/testMinTermOccurrence.csv};
  \addplot [style={ggreen,mark=none}] table [col sep=semicolon,y=Ground Truth 0,x=Min Term Occurrence] {Diagrams/testMinTermOccurrence.csv};
  \addplot [style={ppurple,mark=none}] table [col sep=semicolon,y=Ground Truth Rep 0,x=Min Term Occurrence] {Diagrams/testMinTermOccurrence.csv};
  \addplot [style={tteal,mark=none}] table [col sep=semicolon,y=fMeasure 0,x=Min Term Occurrence] {Diagrams/testMinTermOccurrence.csv};

  \legend{Precision,Recall,F-Measure,Ground truth 0,Ground truth rep 0, F-Measure 0}
  
  \end{axis}
\end{tikzpicture}
  \end{center}
  \caption{Incremental results for minimal term occurrence in collection.}
  \label{diag:topbaseclustersetzioni}
\end{figure}

% MAX TERM RATIO
\begin{figure}[H]
  \begin{center}
\begin{tikzpicture}
  \begin{axis}[
    width  = 0.8*\textwidth,
    height = 4.5cm,
    % major x tick style = transparent,
    xlabel = {Max term ratio},
    ylabel = {Score},
    %xtick = data,
    ymin=0,
    legend cell align=left,
    legend style={
      cells={anchor=east},
      legend pos=outer north east
    }
  ]

  \addplot [style={bblue,mark=none}] table [col sep=semicolon,y=Precision,x=Max Term Ratio] {Diagrams/testMaxTermRatio.csv};
  \addplot [style={rred,mark=none}] table [col sep=semicolon,y=Recall,x=Max Term Ratio] {Diagrams/testMaxTermRatio.csv};
  \addplot [style={oorange,mark=none}] table [col sep=semicolon,y=F-Measure,x=Max Term Ratio] {Diagrams/testMaxTermRatio.csv};
  \addplot [style={ggreen,mark=none}] table [col sep=semicolon,y=Ground Truth 0,x=Max Term Ratio] {Diagrams/testMaxTermRatio.csv};
  \addplot [style={ppurple,mark=none}] table [col sep=semicolon,y=Ground Truth Rep 0,x=Max Term Ratio] {Diagrams/testMaxTermRatio.csv};
  \addplot [style={tteal,mark=none}] table [col sep=semicolon,y=fMeasure 0,x=Max Term Ratio] {Diagrams/testMaxTermRatio.csv};

  \legend{Precision,Recall,F-Measure,Ground truth 0,Ground truth rep 0, F-Measure 0}
  
  \end{axis}
\end{tikzpicture}
  \end{center}
  \caption{Incremental results for max term ratio in collection.}
  \label{diag:topbaseclustersetzioni}
\end{figure}

% Min Limit BC Score
\begin{figure}[H]
  \begin{center}
\begin{tikzpicture}
  \begin{axis}[
    width  = 0.8*\textwidth,
    height = 4.5cm,
    % major x tick style = transparent,
    xlabel = {Min Limit},
    ylabel = {Score},
    %xtick = data,
    ymin=0,
    legend cell align=left,
    legend style={
      cells={anchor=east},
      legend pos=outer north east
    }
  ]

  \addplot [style={bblue,mark=none}] table [col sep=semicolon,y=Precision,x=Min Limit] {Diagrams/testMinLimitBC.csv};
  \addplot [style={rred,mark=none}] table [col sep=semicolon,y=Recall,x=Min Limit] {Diagrams/testMinLimitBC.csv};
  \addplot [style={oorange,mark=none}] table [col sep=semicolon,y=F-Measure,x=Min Limit] {Diagrams/testMinLimitBC.csv};
  \addplot [style={ggreen,mark=none}] table [col sep=semicolon,y=Ground Truth 0,x=Min Limit] {Diagrams/testMinLimitBC.csv};
  \addplot [style={ppurple,mark=none}] table [col sep=semicolon,y=Ground Truth Rep 0,x=Min Limit] {Diagrams/testMinLimitBC.csv};
  \addplot [style={tteal,mark=none}] table [col sep=semicolon,y=fMeasure 0,x=Min Limit] {Diagrams/testMinLimitBC.csv};

  \legend{Precision,Recall,F-Measure,Ground truth 0,Ground truth rep 0, F-Measure 0}
  
  \end{axis}
\end{tikzpicture}
  \end{center}
  \caption{Incremental results for min limit for base cluster score.}
  \label{diag:topbaseclustersetzioni}
\end{figure}

% Max Limit BC Score
\begin{figure}[H]
  \begin{center}
\begin{tikzpicture}
  \begin{axis}[
    width  = 0.8*\textwidth,
    height = 4.5cm,
    % major x tick style = transparent,
    xlabel = {Max Limit},
    ylabel = {Score},
    %xtick = data,
    ymin=0,
    legend cell align=left,
    legend style={
      cells={anchor=east},
      legend pos=outer north east
    }
  ]

  \addplot [style={bblue,mark=none}] table [col sep=semicolon,y=Precision,x=Max Limit] {Diagrams/testMaxLimitBC.csv};
  \addplot [style={rred,mark=none}] table [col sep=semicolon,y=Recall,x=Max Limit] {Diagrams/testMaxLimitBC.csv};
  \addplot [style={oorange,mark=none}] table [col sep=semicolon,y=F-Measure,x=Max Limit] {Diagrams/testMaxLimitBC.csv};
  \addplot [style={ggreen,mark=none}] table [col sep=semicolon,y=Ground Truth 0,x=Max Limit] {Diagrams/testMAxLimitBC.csv};
  \addplot [style={ppurple,mark=none}] table [col sep=semicolon,y=Ground Truth Rep 0,x=Max Limit] {Diagrams/testMaxLimitBC.csv};
  \addplot [style={tteal,mark=none}] table [col sep=semicolon,y=fMeasure 0,x=Max Limit] {Diagrams/testMaxLimitBC.csv};

  \legend{Precision,Recall,F-Measure,Ground truth 0,Ground truth rep 0, F-Measure 0}
  
  \end{axis}
\end{tikzpicture}
  \end{center}
  \caption{Incremental results for max limit for base cluster score.}
  \label{diag:topbaseclustersetzioni}
\end{figure}

% Drop singleton bc test
\begin{figure}[H]
  \begin{center}
\begin{tikzpicture}
  \begin{axis}[
    width  = 0.8*\textwidth,
    height = 4.5cm,
    % major x tick style = transparent,
    ybar=2*\pgflinewidth,
    bar width=8pt,
    ymajorgrids = true,
    ylabel = {Score},
    symbolic x coords={0,1},
    xtick = data,
    scaled y ticks = false,
    enlarge x limits=0.25,
    ymin=0,
    legend cell align=left,
    legend style={
      cells={anchor=east},
      legend pos=outer north east
    }
  ]

  \addplot [style={bblue,fill=bblue,mark=none}] table [col sep=semicolon,y=Precision] {Diagrams/testDropSingletonBC.csv};
  \addplot [style={rred,fill=rred,mark=none}] table [col sep=semicolon,y=Recall] {Diagrams/testDropSingletonBC.csv};
  \addplot [style={oorange,fill=oorange,mark=none}] table [col sep=semicolon,y=F-Measure] {Diagrams/testDropSingletonBC.csv};
  \addplot [style={ggreen,fill=ggreen,mark=none}] table [col sep=semicolon,y=Ground Truth 0] {Diagrams/testDropSingletonBC.csv};
  \addplot [style={ppurple,fill=ppurple,mark=none}] table [col sep=semicolon,y=Ground Truth Rep 0] {Diagrams/testDropSingletonBC.csv};
  \addplot [style={tteal,fill=tteal,mark=none}] table [col sep=semicolon,y=fMeasure 0] {Diagrams/testDropSingletonBC.csv};

  \legend{Precision,Recall,F-Measure,Ground truth 0,Ground truth rep 0, F-Measure 0}
  
  \end{axis}
\end{tikzpicture}
  \end{center}
  \caption{Incremental results for drop singleton base clusters parameter.}
  \label{diag:treetypesetzioni}
\end{figure}

% Drop one word clusters test
\begin{figure}[H]
  \begin{center}
\begin{tikzpicture}
  \begin{axis}[
    width  = 0.8*\textwidth,
    height = 4.5cm,
    % major x tick style = transparent,
    ybar=2*\pgflinewidth,
    bar width=8pt,
    ymajorgrids = true,
    ylabel = {Score},
    symbolic x coords={0,1},
    xtick = data,
    scaled y ticks = false,
    enlarge x limits=0.25,
    ymin=0,
    legend cell align=left,
    legend style={
      cells={anchor=east},
      legend pos=outer north east
    }
  ]

  \addplot [style={bblue,fill=bblue,mark=none}] table [col sep=semicolon,y=Precision] {Diagrams/testDropOneWordClusters.csv};
  \addplot [style={rred,fill=rred,mark=none}] table [col sep=semicolon,y=Recall] {Diagrams/testDropOneWordClusters.csv};
  \addplot [style={oorange,fill=oorange,mark=none}] table [col sep=semicolon,y=F-Measure] {Diagrams/testDropOneWordClusters.csv};
  \addplot [style={ggreen,fill=ggreen,mark=none}] table [col sep=semicolon,y=Ground Truth 0] {Diagrams/testDropOneWordClusters.csv};
  \addplot [style={ppurple,fill=ppurple,mark=none}] table [col sep=semicolon,y=Ground Truth Rep 0] {Diagrams/testDropOneWordClusters.csv};
  \addplot [style={tteal,fill=tteal,mark=none}] table [col sep=semicolon,y=fMeasure 0] {Diagrams/testDropOneWordClusters.csv};

  \legend{Precision,Recall,F-Measure,Ground truth 0,Ground truth rep 0, F-Measure 0}
  
  \end{axis}
\end{tikzpicture}
  \end{center}
  \caption{Incremental results for drop singleton base clusters parameter.}
  \label{diag:treetypesetzioni}
\end{figure}

% Sort descending test
\begin{figure}[H]
  \begin{center}
\begin{tikzpicture}
  \begin{axis}[
    width  = 0.8*\textwidth,
    height = 4.5cm,
    % major x tick style = transparent,
    ybar=2*\pgflinewidth,
    bar width=8pt,
    ymajorgrids = true,
    ylabel = {Score},
    symbolic x coords={0,1},
    xtick = data,
    scaled y ticks = false,
    enlarge x limits=0.25,
    ymin=0,
    legend cell align=left,
    legend style={
      cells={anchor=east},
      legend pos=outer north east
    }
  ]

  \addplot [style={bblue,fill=bblue,mark=none}] table [col sep=semicolon,y=Precision] {Diagrams/testSortDescending.csv};
  \addplot [style={rred,fill=rred,mark=none}] table [col sep=semicolon,y=Recall] {Diagrams/testSortDescending.csv};
  \addplot [style={oorange,fill=oorange,mark=none}] table [col sep=semicolon,y=F-Measure] {Diagrams/testSortDescending.csv};
  \addplot [style={ggreen,fill=ggreen,mark=none}] table [col sep=semicolon,y=Ground Truth 0] {Diagrams/testSortDescending.csv};
  \addplot [style={ppurple,fill=ppurple,mark=none}] table [col sep=semicolon,y=Ground Truth Rep 0] {Diagrams/testSortDescending.csv};
  \addplot [style={tteal,fill=tteal,mark=none}] table [col sep=semicolon,y=fMeasure 0] {Diagrams/testSortDescending.csv};

  \legend{Precision,Recall,F-Measure,Ground truth 0,Ground truth rep 0, F-Measure 0}
  
  \end{axis}
\end{tikzpicture}
  \end{center}
  \caption{Incremental results for sort descending parameter.}
  \label{diag:treetypesetzioni}
\end{figure}

% Text amount
\begin{figure}[H]
  \begin{center}
\begin{tikzpicture}
  \begin{axis}[
    width  = 0.8*\textwidth,
    height = 4.5cm,
    % major x tick style = transparent,
    xlabel = {Article Text Amount},
    xmin=0,
    xmax=1,
    ylabel = {Score},
    %xtick = data,
    ymin=0,
    legend cell align=left,
    legend style={
      cells={anchor=east},
      legend pos=outer north east
    }
  ]

  \addplot [style={bblue,mark=none}] table [col sep=semicolon,y=Precision,x=Article Text Amount] {Diagrams/testArticleTextAmount.csv};
  \addplot [style={rred,mark=none}] table [col sep=semicolon,y=Recall,x=Article Text Amount] {Diagrams/testArticleTextAmount.csv};
  \addplot [style={oorange,mark=none}] table [col sep=semicolon,y=F-Measure,x=Article Text Amount] {Diagrams/testArticleTextAmount.csv};
  \addplot [style={ggreen,mark=none}] table [col sep=semicolon,y=Ground Truth 0,x=Article Text Amount] {Diagrams/testArticleTextAmount.csv};
  \addplot [style={ppurple,mark=none}] table [col sep=semicolon,y=Ground Truth Rep 0,x=Article Text Amount] {Diagrams/testArticleTextAmount.csv};
  \addplot [style={tteal,mark=none}] table [col sep=semicolon,y=fMeasure 0,x=Article Text Amount] {Diagrams/testArticleTextAmount.csv};

  \legend{Precision,Recall,F-Measure,Ground truth 0,Ground truth rep 0, F-Measure 0}
  
  \end{axis}
\end{tikzpicture}
  \end{center}
  \caption{Incremental results for different amount of article text.}
  \label{diag:topbaseclustersetzioni}
\end{figure}

\section{\supervisor parameters}