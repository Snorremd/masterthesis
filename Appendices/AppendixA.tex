%!TEX root = ../Thesis.tex
% Appendix A

\chapter{Incremental test results} % Main appendix title

\label{AppendixA} % For referencing this appendix elsewhere, use \ref{AppendixA}

\lhead{Appendix A. \emph{Incremental Test Results}} % This is for the header on each page - perhaps a shortened title

This appendix contains diagrams of all incremental test results. It serves as a reference for discussion and descriptions in section~\ref{Testing}. The appendix is divided into two main sections. The first contains results from incremental tests using the parameters specified by Etzioni and Oren as base values. The second section contains results using the parameters suggested by \supervisor as base values.

\section{Oren \& Etzioni parameters}

The following results were obtained by applying the parameters used by \citeauthor{Oren1998} as base parameters for the incremental tests. The parameters were collected as closely as possible, but the types of text collected might not be a complete match as different corpora are used.

\begin{lstlisting}[float=t, language=python, label=lst:etzioniparams, caption={Parameter set used in Oren and Etzioni.}]
<parameters>
    <tree_type>(0,0,0)</tree_type> <!-- Suffix tree -->
    <top_base_clusters_amount>500</top_base_clusters_amount>
    <min_term_occurrence_in_collection>4</min_term_occurrence_in_collection>
    <max_term_ratio_in_collection>0.4</max_term_ratio_in_collection>
    <min_limit_for_base_cluster_score>2</min_limit_for_base_cluster_score>
    <max_limit_for_base_cluster_score>7</max_limit_for_base_cluster_score>
    <sort_descending>1</sort_descending>
    <should_drop_singleton_base_clusters>0</should_drop_singleton_base_clusters>
    <should_drop_one_word_clusters>0</should_drop_one_word_clusters>
    <text_amount>0</text_amount>
    <text_types>
    {
    	"FrontPageIntroduction": 1, "FrontPageHeading": 1,
    	"ArticleHeading": 1, "ArticleByline": 1,
    	"ArticleIntroduction": 1, "ArticleText": 0
    }
    </text_types>
    <similarity_measure>
    {
    	"similarity_method": 0,
    	"params": (0.5, 0, 0)
    }
   	</similarity_measure>
</parameters>
\end{lstlisting}

%!TEX root = ../Thesis.tex
% Incremental tests using Oren & Etzioni's parameters as base


\setstretch{1}
% Define bar chart colors
\definecolor{bblue}{HTML}{3366CC}
\definecolor{rred}{HTML}{DC3912}
\definecolor{ggreen}{HTML}{109618}

% TREE TYPE TESTS
\begin{diagram}[H]
  \begin{center}
\begin{tikzpicture}
  \begin{axis}[
    width  = 0.8*\textwidth,
    height = 4.55cm,
    % major x tick style = transparent,
    ybar=2*\pgflinewidth,
    ymajorgrids = true,
    ylabel = {Score},
    xlabel = {Tree types},
    symbolic x coords={Suffix,Midslice,Rangeslice 0.1-1.0,5-slice},
    xtick = data,
    scaled y ticks = false,
    enlarge x limits=0.15,
    ymin=0,
    nodes near coords,
    nodes near coords align={horizontal},
    every node near coord/.append style={font=\tiny,rotate=90,color=black,anchor=west,/pgf/number format/fixed},
    enlarge y limits={upper,value=0.5},
    legend cell align=left,
    legend style={
      cells={anchor=east},
      legend pos=outer north east
    }
  ]

  \addplot [style={rred,fill=rred,mark=none},postaction={pattern=north east lines,pattern color=white}] table [col sep=semicolon,y=Precision 0] {Diagrams/Etzioni/testTrees.csv};
  \addplot [style={bblue,fill=bblue,mark=none},postaction={pattern=north west lines,pattern color=white}] table [col sep=semicolon,y=Ground Truth Rep 0] {Diagrams/Etzioni/testTrees.csv};
  \addplot [style={ggreen,fill=ggreen,mark=none},postaction={pattern=horizontal lines,pattern color=white}] table [col sep=semicolon,y=fMeasure 0] {Diagrams/Etzioni/testTrees.csv};

  \legend{Precision 0,Recall 0, F-Measure 0}
  
  \end{axis}
\end{tikzpicture}
  \end{center}
  \caption{Performance of the \CTC algorithm for different expansion techniques.}
  %\label{diag:treetypesetzioni}
\end{diagram}

% N-SLICE
\begin{diagram}[H]
  \begin{center}
\begin{tikzpicture}
  \begin{axis}[
    width  = 0.8*\textwidth,
    height = 4.55cm,
    % major x tick style = transparent,
    xlabel = {N-Slice length},
    ylabel = {Score},
    %xtick = data,
    ymin=0,
    legend cell align=left,
    legend style={
      cells={anchor=east},
      legend pos=outer north east
    }
  ]

  \addplot+ [style={rred,mark size=1.5}] table [col sep=semicolon,y=Precision 0,x=n-slice] {Diagrams/Etzioni/testNSlices.csv};
  \addplot+ [style={bblue,mark size=1.5}] table [col sep=semicolon,y=Ground Truth Rep 0,x=n-slice] {Diagrams/Etzioni/testNSlices.csv};
  \addplot+ [style={ggreen,mark=triangle*,mark size=1.5}] table [col sep=semicolon,y=fMeasure 0,x=n-slice] {Diagrams/Etzioni/testNSlices.csv};

  \legend{Precision 0,Recall 0, F-Measure 0}
  
  \end{axis}
\end{tikzpicture}
  \end{center}
  \caption{Performance results of the \CTC algorithm for different lengths of n-slice expansion.}
  \label{diag:nslicesetzioni}
\end{diagram}

% RANGE SLICE TEST
\begin{diagram}[H]
  \begin{center}
\begin{tikzpicture}
  \begin{axis}[
    width  = 0.8*\textwidth,
    height = 4.55cm,
    % major x tick style = transparent,
    ybar=2*\pgflinewidth,
    bar width=5pt,
    ymajorgrids = true,
    ylabel = {Score},
    xlabel = {Range slice length},
    symbolic x coords={0.0-1.0,0.1-0.9,0.2-0.8,0.3-0.7,0.4-0.6,0.5-0.5},
    xtick = data,
    scaled y ticks = false,
    enlarge x limits=0.10,
    ymin=0,
    ymax=0.25,
    nodes near coords,
    nodes near coords align={horizontal},
    every node near coord/.append style={font=\tiny,rotate=90,color=black,anchor=west,/pgf/number format/fixed},
    enlarge y limits={upper,value=0.5},
    legend cell align=left,
    legend style={
      cells={anchor=east},
      legend pos=outer north east
    }
  ]

  \addplot [style={rred,fill=rred,mark=none},postaction={pattern=north east lines,pattern color=white}] table [col sep=semicolon,y=Precision 0] {Diagrams/Etzioni/testRangeSlices.csv};
  \addplot [style={bblue,fill=bblue,mark=none},postaction={pattern=north west lines,pattern color=white}] table [col sep=semicolon,y=Ground Truth Rep 0] {Diagrams/Etzioni/testRangeSlices.csv};
  \addplot [style={ggreen,fill=ggreen,mark=none},postaction={pattern=horizontal lines,pattern color=white}] table [col sep=semicolon,y=fMeasure 0] {Diagrams/Etzioni/testRangeSlices.csv};

  \legend{Precision 0,Recall 0, F-Measure 0}
  
  \end{axis}
\end{tikzpicture}
  \end{center}
  \caption{Performance of the \CTC algorithm for different ranges of range-slice expansion.}
  \label{diag:rangelicesetzioni}
\end{diagram}

% NUMBER OF TOP BASE CLUSTERS
\begin{diagram}[H]
  \begin{center}
\begin{tikzpicture}
  \begin{semilogxaxis}[
    width  = 0.8*\textwidth,
    height = 4.55cm,
    % major x tick style = transparent,
    xlabel = {Base cluster amount},
    ylabel = {Score},
    ymin=0,
    % xmin=0,
    legend cell align=left,
    legend style={
      cells={anchor=east},
      legend pos=outer north east
    }
  ]
  \addplot+ [style={rred,mark size=1.5}] table [col sep=semicolon,y=Precision 0,x=Basecluster-amount] {Diagrams/Etzioni/testBaseClusterAmounts.csv};
  \addplot+ [style={bblue,mark size=1.5}] table [col sep=semicolon,y=Ground Truth Rep 0,x=Basecluster-amount] {Diagrams/Etzioni/testBaseClusterAmounts.csv};
  \addplot+ [style={ggreen,mark=triangle*,mark size=1.5}] table [col sep=semicolon,y=fMeasure 0,x=Basecluster-amount] {Diagrams/Etzioni/testBaseClusterAmounts.csv};

  \legend{Precision 0,Recall 0, F-Measure 0}
  
  \end{semilogxaxis}
\end{tikzpicture}
  \end{center}
  \caption{Performance of the \CTC algorithm for different limits on top base clusters amount.}
  %\label{diag:topbaseclustersetzioni}
\end{diagram}

% MIN TERM OCCURRENCE
\begin{diagram}[H]
  \begin{center}
\begin{tikzpicture}
  \begin{semilogxaxis}[
    width  = 0.8*\textwidth,
    height = 4.55cm,
    % major x tick style = transparent,
    xlabel = {Min term occurrence},
    ylabel = {Score},
    %xtick = data,
    % ymin=0,
    % xmin=0,
    % xmax=200,
    legend cell align=left,
    legend style={
      cells={anchor=east},
      legend pos=outer north east
    }
  ]

  \addplot+ [style={rred,mark size=1.5}] table [col sep=semicolon,y=Precision 0,x=Min Term Occurrence] {Diagrams/Etzioni/testMinTermOccurrence.csv};
  \addplot+ [style={bblue,mark size=1.5}] table [col sep=semicolon,y=Ground Truth Rep 0,x=Min Term Occurrence] {Diagrams/Etzioni/testMinTermOccurrence.csv};
  \addplot+ [style={ggreen,mark=triangle*,mark size=1.5}] table [col sep=semicolon,y=fMeasure 0,x=Min Term Occurrence] {Diagrams/Etzioni/testMinTermOccurrence.csv};

  \legend{Precision 0,Recall 0, F-Measure 0}
  
  \end{semilogxaxis}
\end{tikzpicture}
  \end{center}
  \caption{Performance of the \CTC algorithm for different limits on minimal term occurrence in collection.}
  %\label{diag:mintermoccurrenceetzioni}
\end{diagram}

% MAX TERM RATIO
\begin{diagram}[H]
  \begin{center}
\begin{tikzpicture}
  \begin{axis}[
    width  = 0.8*\textwidth,
    height = 4.55cm,
    % major x tick style = transparent,
    xlabel = {Max term ratio},
    ylabel = {Score},
    %xtick = data,
    ymin=0,
    xmin=0.1,
    xmax=1,
    legend cell align=left,
    legend style={
      cells={anchor=east},
      legend pos=outer north east
    }
  ]

  \addplot+ [style={rred,mark size=1.5}] table [col sep=semicolon,y=Precision 0,x=Max Term Ratio] {Diagrams/Etzioni/testMaxTermRatio.csv};
  \addplot+ [style={bblue,mark size=1.5}] table [col sep=semicolon,y=Ground Truth Rep 0,x=Max Term Ratio] {Diagrams/Etzioni/testMaxTermRatio.csv};
  \addplot+ [style={ggreen,mark=triangle*,mark size=1.5}] table [col sep=semicolon,y=fMeasure 0,x=Max Term Ratio] {Diagrams/Etzioni/testMaxTermRatio.csv};

  \legend{Precision 0,Recall 0, F-Measure 0}
  
  \end{axis}
\end{tikzpicture}
  \end{center}
  \caption{Performance of the \CTC algorithm for different limits on max term ratio in collection.}
  \label{diag:maxtermratioetzioni}
\end{diagram}

% Min Limit BC Score
\begin{diagram}[H]
  \begin{center}
\begin{tikzpicture}
  \begin{axis}[
    width  = 0.8*\textwidth,
    height = 4.55cm,
    % major x tick style = transparent,
    xlabel = {Min Limit BC Score},
    ylabel = {Score},
    %xtick = data,
    ymin=0,
    xmin=0,
    xmax=15,
    legend cell align=left,
    legend style={
      cells={anchor=east},
      legend pos=outer north east
    }
  ]

  \addplot+ [style={rred,mark size=1.5}] table [col sep=semicolon,y=Precision 0,x=Min Limit] {Diagrams/Etzioni/testMinLimitBC.csv};
  \addplot+ [style={bblue,mark size=1.5}] table [col sep=semicolon,y=Ground Truth Rep 0,x=Min Limit] {Diagrams/Etzioni/testMinLimitBC.csv};
  \addplot+ [style={ggreen,mark=triangle*,mark size=1.5}] table [col sep=semicolon,y=fMeasure 0,x=Min Limit] {Diagrams/Etzioni/testMinLimitBC.csv};

  \legend{Precision 0,Recall 0, F-Measure 0}
  
  \end{axis}
\end{tikzpicture}
  \end{center}
  \caption{Performance of the \CTC algorithm for different min limit values for base cluster score with unbounded max limit (max limit = length of longest label).}
  %\label{diag:minlimitbcscoreetzioni}
\end{diagram}

% Max Limit BC Score
\begin{diagram}[H]
  \begin{center}
\begin{tikzpicture}
  \begin{axis}[
    width  = 0.8*\textwidth,
    height = 4.55cm,
    % major x tick style = transparent,
    xlabel = {Max Limit BC Score},
    ylabel = {Score},
    xtick = data,
    ymin=0,
    xmin=3,
    xmax=15,
    legend cell align=left,
    legend style={
      cells={anchor=east},
      legend pos=outer north east
    }
  ]

  \addplot+ [style={rred,mark size=1.5}] table [col sep=semicolon,y=Precision 0,x=Max Limit] {Diagrams/Etzioni/testMaxLimitBC.csv};
  \addplot+ [style={bblue,mark size=1.5}] table [col sep=semicolon,y=Ground Truth Rep 0,x=Max Limit] {Diagrams/Etzioni/testMaxLimitBC.csv};
  \addplot+ [style={ggreen,mark=triangle*,mark size=1.5}] table [col sep=semicolon,y=fMeasure 0,x=Max Limit] {Diagrams/Etzioni/testMaxLimitBC.csv};

  \legend{Precision 0,Recall 0, F-Measure 0}
  
  \end{axis}
\end{tikzpicture}
  \end{center}
  \caption{Performance of the \CTC algorithm for different max limit values for base cluster score. Min limit set to \protect\citeauthor{Oren1998} default.}
  \label{diag:maxlimitbcscoreetzioni}
\end{diagram}

% Max Limit BC Score best min value
\begin{diagram}[H]
  \begin{center}
\begin{tikzpicture}
  \begin{axis}[
    width  = 0.8*\textwidth,
    height = 4.55cm,
    % major x tick style = transparent,
    xlabel = {Max Limit BC Score},
    ylabel = {Score},
    xtick = data,
    ymin=0,
    xmin=9,
    xmax=15,
    legend cell align=left,
    legend style={
      cells={anchor=east},
      legend pos=outer north east
    }
  ]

  \addplot+ [style={rred,mark size=1.5}] table [col sep=semicolon,y=Precision 0,x=Max Limit] {Diagrams/Etzioni/testMaxLimitBC2.csv};
  \addplot+ [style={bblue,mark size=1.5}] table [col sep=semicolon,y=Ground Truth Rep 0,x=Max Limit] {Diagrams/Etzioni/testMaxLimitBC2.csv};
  \addplot+ [style={ggreen,mark=triangle*,mark size=1.5}] table [col sep=semicolon,y=fMeasure 0,x=Max Limit] {Diagrams/Etzioni/testMaxLimitBC2.csv};

  \legend{Precision 0,Recall 0, F-Measure 0}
  
  \end{axis}
\end{tikzpicture}
  \end{center}
  \caption{Performance of the \CTC algorithm for different max limit values for base cluster score. Min limit set to 8, the best min limit from incremental test on min limit for base cluster score.}
  \label{diag:maxlimitbcscoreetzioni2}
\end{diagram}

% Drop singleton bc test
\begin{diagram}[H]
  \begin{center}
\begin{tikzpicture}
  \begin{axis}[
    width  = 0.8*\textwidth,
    height = 4.55cm,
    % major x tick style = transparent,
    ybar=2*\pgflinewidth,
    bar width=8pt,
    ymajorgrids = true,
    ylabel = {Score},
    xlabel = {Drop singleton base clusters?},
    symbolic x coords={0,1},
    xtick = data,
    scaled y ticks = false,
    enlarge x limits=0.25,
    ymin=0,
    nodes near coords,
    nodes near coords align={horizontal},
    every node near coord/.append style={font=\tiny,rotate=90,color=black,anchor=west,/pgf/number format/fixed},
    enlarge y limits={upper,value=0.5},
    legend cell align=left,
    legend style={
      cells={anchor=east},
      legend pos=outer north east
    }
  ]

  \addplot [style={rred,fill=rred,mark=none},postaction={pattern=north east lines,pattern color=white}] table [col sep=semicolon,y=Precision 0] {Diagrams/Etzioni/testDropSingletonBC.csv};
  \addplot [style={bblue,fill=bblue,mark=none},postaction={pattern=north west lines,pattern color=white}] table [col sep=semicolon,y=Ground Truth Rep 0] {Diagrams/Etzioni/testDropSingletonBC.csv};
  \addplot [style={ggreen,fill=ggreen,mark=none},postaction={pattern=horizontal lines,pattern color=white}] table [col sep=semicolon,y=fMeasure 0] {Diagrams/Etzioni/testDropSingletonBC.csv};

  \legend{Precision 0,Recall 0, F-Measure 0}
  
  \end{axis}
\end{tikzpicture}
  \end{center}
  \caption{Performance of the \CTC algorithm for exclusion and inclusion of singleton base clusters.}
  \label{diag:dropsingletonbcetzioni}
\end{diagram}

% Drop one word clusters test
\begin{diagram}[H]
  \begin{center}
\begin{tikzpicture}
  \begin{axis}[
    width  = 0.8*\textwidth,
    height = 4.55cm,
    % major x tick style = transparent,
    ybar=2*\pgflinewidth,
    bar width=8pt,
    ymajorgrids = true,
    ylabel = {Score},
    xlabel = {Drop one word clusters?},
    symbolic x coords={0,1},
    xtick = data,
    scaled y ticks = false,
    enlarge x limits=0.25,
    ymin=0,
    nodes near coords,
    nodes near coords align={horizontal},
    every node near coord/.append style={font=\tiny,rotate=90,color=black,anchor=west,/pgf/number format/fixed},
    enlarge y limits={upper,value=0.5},
    legend cell align=left,
    legend style={
      cells={anchor=east},
      legend pos=outer north east
    }
  ]

  \addplot [style={rred,fill=rred,mark=none},postaction={pattern=north east lines,pattern color=white}] table [col sep=semicolon,y=Precision 0] {Diagrams/Etzioni/testDropOneWordClusters.csv};
  \addplot [style={bblue,fill=bblue,mark=none},postaction={pattern=north west lines,pattern color=white}] table [col sep=semicolon,y=Ground Truth Rep 0] {Diagrams/Etzioni/testDropOneWordClusters.csv};
  \addplot [style={ggreen,fill=ggreen,mark=none},postaction={pattern=horizontal lines,pattern color=white}] table [col sep=semicolon,y=fMeasure 0] {Diagrams/Etzioni/testDropOneWordClusters.csv};

  \legend{Precision 0,Recall 0, F-Measure 0}
  
  \end{axis}
\end{tikzpicture}
  \end{center}
  \caption{Performance of the \CTC algorithm on exclusion and inclusion of one word clusters.}
  \label{diag:droponewordclustersetzioni}
\end{diagram}

% Sort descending test
\begin{diagram}[H]
  \begin{center}
\begin{tikzpicture}
  \begin{axis}[
    width  = 0.8*\textwidth,
    height = 4.55cm,
    % major x tick style = transparent,
    ybar=2*\pgflinewidth,
    bar width=8pt,
    ymajorgrids = true,
    ylabel = {Score},
    xlabel = {Sort descending?},
    symbolic x coords={0,1},
    xtick = data,
    scaled y ticks = false,
    enlarge x limits=0.25,
    ymin=0,
    nodes near coords,
    nodes near coords align={horizontal},
    every node near coord/.append style={font=\tiny,rotate=90,color=black,anchor=west,/pgf/number format/fixed},
    enlarge y limits={upper,value=0.5},
    legend cell align=left,
    legend style={
      cells={anchor=east},
      legend pos=outer north east
    }
  ]

  \addplot [style={rred,fill=rred,mark=none},postaction={pattern=north east lines,pattern color=white}] table [col sep=semicolon,y=Precision 0] {Diagrams/Etzioni/testSortDescending.csv};
  \addplot [style={bblue,fill=bblue,mark=none},postaction={pattern=north west lines,pattern color=white}] table [col sep=semicolon,y=Ground Truth Rep 0] {Diagrams/Etzioni/testSortDescending.csv};
  \addplot [style={ggreen,fill=ggreen,mark=none},postaction={pattern=horizontal lines,pattern color=white}] table [col sep=semicolon,y=fMeasure 0] {Diagrams/Etzioni/testSortDescending.csv};

  \legend{Precision 0,Recall 0, F-Measure 0}
  
  \end{axis}
\end{tikzpicture}
  \end{center}
  \caption{Performance of the \CTC algorithm when base clusters are sorted in descending and acending order.}
  %\label{diag:sortdescendingetzioni}
\end{diagram}

% Text amount
\begin{diagram}[H]
  \begin{center}
\begin{tikzpicture}
  \begin{axis}[
    width  = 0.8*\textwidth,
    height = 4.55cm,
    % major x tick style = transparent,
    xlabel = {Article Text Amount},
    xmin=0,
    xmax=1,
    ylabel = {Score},
    %xtick = data,
    ymin=0,
    legend cell align=left,
    legend style={
      cells={anchor=east},
      legend pos=outer north east
    }
  ]

  \addplot+ [style={rred,mark size=1.5}] table [col sep=semicolon,y=Precision 0,x=Article Text Amount] {Diagrams/Etzioni/testArticleTextAmount.csv};
  \addplot+ [style={bblue,mark size=1.5}] table [col sep=semicolon,y=Ground Truth Rep 0,x=Article Text Amount] {Diagrams/Etzioni/testArticleTextAmount.csv};
  \addplot+ [style={ggreen,mark=triangle*,mark size=1.5}] table [col sep=semicolon,y=fMeasure 0,x=Article Text Amount] {Diagrams/Etzioni/testArticleTextAmount.csv};

  \legend{Precision 0,Recall 0, F-Measure 0}
  
  \end{axis}
\end{tikzpicture}
  \end{center}
  \caption{Performance of the \CTC algorithm for different amounts of article text.}
  \label{diag:textamountetzioni}
\end{diagram}

% TEXT TYPE TESTS
\begin{diagram}[H]
  \begin{center}
\begin{tikzpicture}
  \begin{axis}[
    width  = 0.8*\textwidth,
    height = 4.55cm,
    % major x tick style = transparent,
    ybar=2*\pgflinewidth,
    bar width=6pt,
    ymajorgrids = true,
    ylabel = {Score},
    symbolic x coords={All,Frontpage,Article sans bread text,Article with bread text,Article text},
    x tick label style={font=\small,text width=1.7cm,align=center},
    xtick = data,
    xlabel = {Text types included},
    scaled y ticks = false,
    enlarge x limits=0.10,
    ymin=0,
    nodes near coords,
    nodes near coords align={horizontal},
    every node near coord/.append style={font=\tiny,rotate=90,color=black,anchor=west,/pgf/number format/fixed},
    enlarge y limits={upper,value=0.5},
    legend cell align=left,
    legend style={
      cells={anchor=east},
      legend pos=outer north east
    }
  ]

  \addplot [style={rred,fill=rred,mark=none},postaction={pattern=north east lines,pattern color=white}] table [col sep=semicolon,y=Precision 0] {Diagrams/Etzioni/testTextTypes.csv};
  \addplot [style={bblue,fill=bblue,mark=none},postaction={pattern=north west lines,pattern color=white}] table [col sep=semicolon,y=Ground Truth Rep 0] {Diagrams/Etzioni/testTextTypes.csv};
  \addplot [style={ggreen,fill=ggreen,mark=none},postaction={pattern=horizontal lines,pattern color=white}] table [col sep=semicolon,y=fMeasure 0] {Diagrams/Etzioni/testTextTypes.csv};

  \legend{Precision 0,Recall 0, F-Measure 0}
  
  \end{axis}
\end{tikzpicture}
  \end{center}
  \caption{Performance of the \CTC algorithm for inclusion of different types of texts.}
  %\label{diag:texttypesetzioni}
\end{diagram}

% SIMILARITY METHODS TESTS
\begin{diagram}[H]
  \begin{center}
\begin{tikzpicture}
  \begin{axis}[
    width  = 0.8*\textwidth,
    height = 4.55cm,
    % major x tick style = transparent,
    ybar=2*\pgflinewidth,
    bar width=8pt,
    ymajorgrids = true,
    ylabel = {Score},
    xlabel = {Similarity methods},
    symbolic x coords={Etzioni,Jaccard,Cosine,Amendment1C},
    xtick = data,
    scaled y ticks = false,
    enlarge x limits=0.20,
    ymin=0,
    nodes near coords,
    nodes near coords align={horizontal},
    every node near coord/.append style={font=\tiny,rotate=90,color=black,anchor=west, /pgf/number format/fixed},
    enlarge y limits={upper,value=0.5},
    legend cell align=left,
    legend style={
      cells={anchor=east},
      legend pos=outer north east
    }
  ]

  \addplot [style={rred,fill=rred,mark=none},postaction={pattern=north east lines,pattern color=white}] table [col sep=semicolon,y=Precision 0] {Diagrams/Etzioni/testSimilarityMethods.csv};
  \addplot [style={bblue,fill=bblue,mark=none},postaction={pattern=north west lines,pattern color=white}] table [col sep=semicolon,y=Ground Truth Rep 0] {Diagrams/Etzioni/testSimilarityMethods.csv};
  \addplot [style={ggreen,fill=ggreen,mark=none},postaction={pattern=horizontal lines,pattern color=white}] table [col sep=semicolon,y=fMeasure 0] {Diagrams/Etzioni/testSimilarityMethods.csv};

  \legend{Precision 0,Recall 0, F-Measure 0}
  
  \end{axis}
\end{tikzpicture}
  \end{center}
  \caption{Performance of the \CTC algorithm for different similarity methods.}
  \label{diag:similaritymethodsetzioni}
\end{diagram}

% Etzioni THRESHOLD
\begin{diagram}[H]
  \begin{center}
\begin{tikzpicture}
  \begin{axis}[
    % Sizing
    width  = 0.8*\textwidth,
    height = 4.55cm,
    % Data
    xlabel = {Etzioni Similarity Threshold},
    xmin=0,
    xmax=1,
    ymin=0,
    % Labeling
    ylabel = {Score},
    legend cell align=left,
    legend style={
      cells={anchor=east},
      legend pos=outer north east
    }
  ]

  \addplot+ [style={rred,mark size=1.5}] table [col sep=semicolon,y=Precision 0,x=Threshold] {Diagrams/Etzioni/testEtzioniSimilarity.csv};
  \addplot+ [style={bblue,mark size=1.5}] table [col sep=semicolon,y=Ground Truth Rep 0,x=Threshold] {Diagrams/Etzioni/testEtzioniSimilarity.csv};
  \addplot+ [style={ggreen,mark=triangle*,mark size=1.5}] table [col sep=semicolon,y=fMeasure 0,x=Threshold] {Diagrams/Etzioni/testEtzioniSimilarity.csv};

  \legend{Precision 0,Recall 0, F-Measure 0}
  
  \end{axis}
\end{tikzpicture}
  \end{center}
  \caption{Performance of the \CTC algorithm for different Etzioni similarity thresholds.}
  \label{diag:etzionithresholdetzioni}
\end{diagram}

% JACCARD THRESHOLD
\begin{diagram}[H]
  \begin{center}
\begin{tikzpicture}
  \begin{axis}[
    % Sizing
    width  = 0.8*\textwidth,
    height = 4.55cm,
    % Data
    xlabel = {Jaccard Coefficient Threshold},
    xmin=0,
    xmax=1,
    ymin=0,
    % Labeling
    ylabel = {Score},
    legend cell align=left,
    legend style={
      cells={anchor=east},
      legend pos=outer north east
    }
  ]

  \addplot+ [style={rred,mark size=1.5}] table [col sep=semicolon,y=Precision 0,x=Threshold] {Diagrams/Etzioni/testJaccardSimilarity.csv};
  \addplot+ [style={bblue,mark size=1.5}] table [col sep=semicolon,y=Ground Truth Rep 0,x=Threshold] {Diagrams/Etzioni/testJaccardSimilarity.csv};
  \addplot+ [style={ggreen,mark=triangle*,mark size=1.5}] table [col sep=semicolon,y=fMeasure 0,x=Threshold] {Diagrams/Etzioni/testJaccardSimilarity.csv};

  \legend{Precision 0,Recall 0, F-Measure 0}
  
  \end{axis}
\end{tikzpicture}
  \end{center}
  \caption{Performance of the \CTC algorithm for different Jaccard Coefficient thresholds.}
  \label{diag:jaccardthresholdetzioni}
\end{diagram}

% COSINE THRESHOLD
\begin{diagram}[H]
  \begin{center}
\begin{tikzpicture}
  \begin{axis}[
    % Sizing
    width  = 0.8*\textwidth,
    height = 4.55cm,
    % Data
    xlabel = {Cosine Threshold},
    xmin=0,
    xmax=1,
    ymin=0,
    % Labeling
    ylabel = {Score},
    legend cell align=left,
    legend style={
      cells={anchor=east},
      legend pos=outer north east
    }
  ]

  \addplot+ [style={rred,mark size=1.5}] table [col sep=semicolon,y=Precision 0,x=Cosine Threshold] {Diagrams/Etzioni/testCosineSimilarity.csv};
  \addplot+ [style={bblue,mark size=1.5}] table [col sep=semicolon,y=Ground Truth Rep 0,x=Cosine Threshold] {Diagrams/Etzioni/testCosineSimilarity.csv};
  \addplot+ [style={ggreen,mark=triangle*,mark size=1.5}] table [col sep=semicolon,y=fMeasure 0,x=Cosine Threshold] {Diagrams/Etzioni/testCosineSimilarity.csv};

  \legend{Precision 0,Recall 0, F-Measure 0}
  
  \end{axis}
\end{tikzpicture}
  \end{center}
  \caption{Performance of the \CTC algorithm for different Cosine Similarity thresholds.}
  \label{diag:cosinethresholdetzioni}
\end{diagram}

% Average corpus frequency limit
\begin{diagram}[H]
  \begin{center}
\begin{tikzpicture}
  \begin{axis}[
    width  = 0.8*\textwidth,
    height = 4.55cm,
    % major x tick style = transparent,
    xlabel = {Avg corpus frequency limit},
    ylabel = {Score},
    %xtick = data,
    ymin=0,
    xmin=0,
    xmax=500,
    legend cell align=left,
    legend style={
      cells={anchor=east},
      legend pos=outer north east
    }
  ]

  \addplot+ [style={rred,mark size=1.5}] table [col sep=semicolon,y=Precision 0,x=Avg CF limit] {Diagrams/Etzioni/testAmendment1CSimilarityAvgCF.csv};
  \addplot+ [style={bblue,mark size=1.5}] table [col sep=semicolon,y=Ground Truth Rep 0,x=Avg CF limit] {Diagrams/Etzioni/testAmendment1CSimilarityAvgCF.csv};
  \addplot+ [style={ggreen,mark=triangle*,mark size=1.5}] table [col sep=semicolon,y=fMeasure 0,x=Avg CF limit] {Diagrams/Etzioni/testAmendment1CSimilarityAvgCF.csv};

  \legend{Precision 0,Recall 0, F-Measure 0}
  
  \end{axis}
\end{tikzpicture}
  \end{center}
  \caption{Performance of the \CTC algorithm for different limits on max average corpus frequency in Amendment1C.}
  \label{diag:avgcfamendment1etzioni}
\end{diagram}

% Base cluster intersect min limit
\begin{diagram}[H]
  \begin{center}
\begin{tikzpicture}
  \begin{axis}[
    width  = 0.8*\textwidth,
    height = 4.55cm,
    % major x tick style = transparent,
    xlabel = {Min label intersect limit},
    ylabel = {Score},
    %xtick = data,
    ymin=0,
    xmin=0,
    xmax=50,
    legend cell align=left,
    legend style={
      cells={anchor=east},
      legend pos=outer north east
    }
  ]

  \addplot+ [style={rred,mark size=1.5}] table [col sep=semicolon,y=Precision 0,x=Min intersect limit] {Diagrams/Etzioni/testAmendment1CSimilarityIntersect.csv};
  \addplot+ [style={bblue,mark size=1.5}] table [col sep=semicolon,y=Ground Truth Rep 0,x=Min intersect limit] {Diagrams/Etzioni/testAmendment1CSimilarityIntersect.csv};
  \addplot+ [style={ggreen,mark=triangle*,mark size=1.5}] table [col sep=semicolon,y=fMeasure 0,x=Min intersect limit] {Diagrams/Etzioni/testAmendment1CSimilarityIntersect.csv};

  \legend{Precision 0,Recall 0, F-Measure 0}
  
  \end{axis}
\end{tikzpicture}
  \end{center}
  \caption{Performance of the \CTC algorithm for different minimum limits on base cluster label intersect in Amendment1C.}
  \label{diag:minintersectamendment1cetzioni}
\end{diagram}

\section{\supervisor parameters}

The following results were obtained by applying the LII groups parameters as base parameters for the incremental tests.

\begin{lstlisting}[float=t, language=python, label=lst:richardparams, caption={Parameter set used by \protect\supervisor.}]
<parameters>
    <tree_type>(1,0,0)</tree_type>
    <top_base_clusters_amount>5000</top_base_clusters_amount>
    <min_term_occurrence_in_collection>6</min_term_occurrence_in_collection>
    <max_term_ratio_in_collection>0.6</max_term_ratio_in_collection>
    <min_limit_for_base_cluster_score>2</min_limit_for_base_cluster_score>
    <max_limit_for_base_cluster_score>7</max_limit_for_base_cluster_score>
    <sort_descending>0</sort_descending>
    <should_drop_singleton_base_clusters>0</should_drop_singleton_base_clusters>
    <should_drop_one_word_clusters>1</should_drop_one_word_clusters>
    <text_amount>0.15</text_amount>
    <text_types>
        {"FrontPageIntroduction": 0,
         "FrontPageHeading": 0,
         "ArticleHeading": 1,
         "ArticleByline": 1,
         "ArticleIntroduction": 1,
         "ArticleText": 0}
    </text_types>
    <similarity_measure>
        {"similarity_method": 0,
         "params": (0.5, 0, 0)}
    </similarity_measure>
</parameters>
\end{lstlisting}

%!TEX root = ../Thesis.tex
% Incremental tests using Oren & Richard's parameters as base


\setstretch{1}
% Define bar chart colors
\definecolor{bblue}{HTML}{3366CC}
\definecolor{rred}{HTML}{DC3912}
\definecolor{ggreen}{HTML}{109618}

% TREE TYPE TESTS
\begin{diagram}[H]
  \begin{center}
\begin{tikzpicture}
  \begin{axis}[
    width  = 0.8*\textwidth,
    height = 4.55cm,
    % major x tick style = transparent,
    ybar=2*\pgflinewidth,
    ymajorgrids = true,
    ylabel = {Score},
    xlabel = {Tree types},
    symbolic x coords={Suffix,Midslice,Rangeslice 0.1-1.0,5-slice},
    xtick = data,
    scaled y ticks = false,
    enlarge x limits=0.15,
    ymin=0,
    nodes near coords,
    nodes near coords align={horizontal},
    every node near coord/.append style={font=\tiny,rotate=90,color=black,anchor=west,/pgf/number format/fixed},
    enlarge y limits={upper,value=0.5},
    legend cell align=left,
    legend style={
      cells={anchor=east},
      legend pos=outer north east
    }
  ]

  \addplot [style={rred,fill=rred,mark=none},postaction={pattern=north east lines,pattern color=white}] table [col sep=semicolon,y=Ground Truth 0] {Diagrams/Richard/testTrees.csv};
  \addplot [style={bblue,fill=bblue,mark=none},postaction={pattern=north west lines,pattern color=white}] table [col sep=semicolon,y=Ground Truth Rep 0] {Diagrams/Richard/testTrees.csv};
  \addplot [style={ggreen,fill=ggreen,mark=none},postaction={pattern=horizontal lines,pattern color=white}] table [col sep=semicolon,y=fMeasure 0] {Diagrams/Richard/testTrees.csv};

  \legend{Precision 0,Recall 0, F-Measure 0}
  
  \end{axis}
\end{tikzpicture}
  \end{center}
  \caption{Performance of the \CTC algorithm for different expansion techniques.}
  \label{diag:treetypesrichard}
\end{diagram}

% N-SLICE
\begin{diagram}[H]
  \begin{center}
\begin{tikzpicture}
  \begin{axis}[
    width  = 0.8*\textwidth,
    height = 4.55cm,
    % major x tick style = transparent,
    xlabel = {N-Slice length},
    ylabel = {Score},
    % xtick = data,
    ymin=0,
    legend cell align=left,
    legend style={
      cells={anchor=east},
      legend pos=outer north east
    }
  ]

  \addplot+ [style={rred,mark size=1.5}] table [col sep=semicolon,y=Ground Truth 0,x=n-slice] {Diagrams/Richard/testNSlices.csv};
  \addplot+ [style={bblue,mark size=1.5}] table [col sep=semicolon,y=Ground Truth Rep 0,x=n-slice] {Diagrams/Richard/testNSlices.csv};
  \addplot+ [style={ggreen,mark=triangle*,mark size=1.5}] table [col sep=semicolon,y=fMeasure 0,x=n-slice] {Diagrams/Richard/testNSlices.csv};

  \legend{Precision 0,Recall 0, F-Measure 0}
  
  \end{axis}
\end{tikzpicture}
  \end{center}
  \caption{Performance results of the \CTC algorithm for different lengths of n-slice expansion.}
  \label{diag:nslicesrichard}
\end{diagram}

% RANGE SLICE TEST
\begin{diagram}[H]
  \begin{center}
\begin{tikzpicture}
  \begin{axis}[
    width  = 0.8*\textwidth,
    height = 4.55cm,
    % major x tick style = transparent,
    ybar=2*\pgflinewidth,
    bar width=5pt,
    ymajorgrids = true,
    ylabel = {Score},
    xlabel = {Range slice length},
    symbolic x coords={0.0-1.0,0.1-0.9,0.2-0.8,0.3-0.7,0.4-0.6,0.5-0.5},
    xtick = data,
    scaled y ticks = false,
    enlarge x limits=0.10,
    ymin=0,
    % ymax=0.25,
    nodes near coords,
    nodes near coords align={horizontal},
    every node near coord/.append style={font=\tiny,rotate=90,color=black,anchor=west,/pgf/number format/fixed},
    enlarge y limits={upper,value=0.5},
    legend cell align=left,
    legend style={
      cells={anchor=east},
      legend pos=outer north east
    }
  ]

  \addplot [style={rred,fill=rred,mark=none},postaction={pattern=north east lines,pattern color=white}] table [col sep=semicolon,y=Ground Truth 0] {Diagrams/Richard/testRangeSlices.csv};
  \addplot [style={bblue,fill=bblue,mark=none},postaction={pattern=north west lines,pattern color=white}] table [col sep=semicolon,y=Ground Truth Rep 0] {Diagrams/Richard/testRangeSlices.csv};
  \addplot [style={ggreen,fill=ggreen,mark=none},postaction={pattern=horizontal lines,pattern color=white}] table [col sep=semicolon,y=fMeasure 0] {Diagrams/Richard/testRangeSlices.csv};

  \legend{Precision 0,Recall 0, F-Measure 0}
  
  \end{axis}
\end{tikzpicture}
  \end{center}
  \caption{Performance of the \CTC algorithm for different ranges of range-slice expansion.}
  \label{diag:rangelicesrichard}
\end{diagram}

% % RANGE SLICE TEST
% \begin{diagram}[H]
%   \begin{center}
% \begin{tikzpicture}
%   \begin{axis}[
%     width  = 0.8*\textwidth,
%     height = 4.55cm,
%     % major x tick style = transparent,
%     ybar=2*\pgflinewidth,
%     bar width=5pt,
%     ymajorgrids = true,
%     ylabel = {Score},
%     xlabel = {Range slice length},
%     symbolic x coords={0.0-1.0,0.1-0.9,0.2-0.8,0.3-0.7,0.4-0.6,0.5-0.5},
%     xtick = data,
%     scaled y ticks = false,
%     enlarge x limits=0.10,
%     ymin=0,
%     ymax=0.25,
%     nodes near coords,
%     nodes near coords align={horizontal},
%     every node near coord/.append style={font=\tiny,rotate=90,color=black,anchor=west,/pgf/number format/fixed},
%     enlarge y limits={upper,value=0.5},
%     legend cell align=left,
%     legend style={
%       cells={anchor=east},
%       legend pos=outer north east
%     }
%   ]

%   \addplot [style={rred,fill=rred,mark=none},postaction={pattern=north east lines,pattern color=white}] table [col sep=semicolon,y=Ground Truth 0] {Diagrams/Richard/testRangeSlices.csv};
%   \addplot [style={bblue,fill=bblue,mark=none},postaction={pattern=north west lines,pattern color=white}] table [col sep=semicolon,y=Ground Truth Rep 0] {Diagrams/Richard/testRangeSlices.csv};
%   \addplot [style={ggreen,fill=ggreen,mark=none},postaction={pattern=horizontal lines,pattern color=white}] table [col sep=semicolon,y=fMeasure 0] {Diagrams/Richard/testRangeSlices.csv};

%   \legend{Precision 0,Recall 0, F-Measure 0}
  
%   \end{axis}
% \end{tikzpicture}
%   \end{center}
%   \caption{Performance of the \CTC algorithm for different ranges of range-slice expansion.}
%   \label{diag:rangelicesrichard}
% \end{diagram}

% NUMBER OF TOP BASE CLUSTERS
\begin{diagram}[H]
  \begin{center}
\begin{tikzpicture}
  \begin{semilogxaxis}[
    width  = 0.8*\textwidth,
    height = 4.55cm,
    % major x tick style = transparent,
    xlabel = {Base cluster amount},
    ylabel = {Score},
    ymin=0,
    legend cell align=left,
    legend style={
      cells={anchor=east},
      legend pos=outer north east
    }
  ]
  \addplot+ [style={rred,mark size=1.5}] table [col sep=semicolon,y=Ground Truth 0,x=Basecluster-amount] {Diagrams/Richard/testBaseClusterAmounts.csv};
  \addplot+ [style={bblue,mark size=1.5}] table [col sep=semicolon,y=Ground Truth Rep 0,x=Basecluster-amount] {Diagrams/Richard/testBaseClusterAmounts.csv};
  \addplot+ [style={ggreen,mark=triangle*,mark size=1.5}] table [col sep=semicolon,y=fMeasure 0,x=Basecluster-amount] {Diagrams/Richard/testBaseClusterAmounts.csv};

  \legend{Precision 0,Recall 0, F-Measure 0}
  
  \end{semilogxaxis}
\end{tikzpicture}
  \end{center}
  \caption{Performance of the \CTC algorithm for different limits on top base clusters amount.}
  \label{diag:topbaseclustersrichard}
\end{diagram}

% MIN TERM OCCURRENCE
\begin{diagram}[H]
  \begin{center}
\begin{tikzpicture}
  \begin{semilogxaxis}[
    width  = 0.8*\textwidth,
    height = 4.55cm,
    % major x tick style = transparent,
    xlabel = {Min term occurrence},
    ylabel = {Score},
    %xtick = data,
    % ymin=0,
    % xmin=0,
    % xmax=200,
    legend cell align=left,
    legend style={
      cells={anchor=east},
      legend pos=outer north east
    }
  ]

  \addplot+ [style={rred,mark size=1.5}] table [col sep=semicolon,y=Ground Truth 0,x=Min Term Occurrence] {Diagrams/Richard/testMinTermOccurrence.csv};
  \addplot+ [style={bblue,mark size=1.5}] table [col sep=semicolon,y=Ground Truth Rep 0,x=Min Term Occurrence] {Diagrams/Richard/testMinTermOccurrence.csv};
  \addplot+ [style={ggreen,mark=triangle*,mark size=1.5}] table [col sep=semicolon,y=fMeasure 0,x=Min Term Occurrence] {Diagrams/Richard/testMinTermOccurrence.csv};

  \legend{Precision 0,Recall 0, F-Measure 0}
  
  \end{semilogxaxis}
\end{tikzpicture}
  \end{center}
  \caption{Performance of the \CTC algorithm for different limits on minimal term occurrence in collection.}
  %\label{diag:mintermoccurrencerichard}
\end{diagram}

% MAX TERM RATIO
\begin{diagram}[H]
  \begin{center}
\begin{tikzpicture}
  \begin{axis}[
    width  = 0.8*\textwidth,
    height = 4.55cm,
    % major x tick style = transparent,
    xlabel = {Max term ratio},
    ylabel = {Score},
    %xtick = data,
    ymin=0,
    xmin=0.1,
    xmax=1,
    legend cell align=left,
    legend style={
      cells={anchor=east},
      legend pos=outer north east
    }
  ]

  \addplot+ [style={rred,mark size=1.5}] table [col sep=semicolon,y=Ground Truth 0,x=Max Term Ratio] {Diagrams/Richard/testMaxTermRatio.csv};
  \addplot+ [style={bblue,mark size=1.5}] table [col sep=semicolon,y=Ground Truth Rep 0,x=Max Term Ratio] {Diagrams/Richard/testMaxTermRatio.csv};
  \addplot+ [style={ggreen,mark=triangle*,mark size=1.5}] table [col sep=semicolon,y=fMeasure 0,x=Max Term Ratio] {Diagrams/Richard/testMaxTermRatio.csv};

  \legend{Precision 0,Recall 0, F-Measure 0}
  
  \end{axis}
\end{tikzpicture}
  \end{center}
  \caption{Performance of the \CTC algorithm for different limits on max term ratio in collection.}
  \label{diag:maxtermratiorichard}
\end{diagram}

% Min Limit BC Score
\begin{diagram}[H]
  \begin{center}
\begin{tikzpicture}
  \begin{axis}[
    width  = 0.8*\textwidth,
    height = 4.55cm,
    % major x tick style = transparent,
    xlabel = {Min Limit BC Score},
    ylabel = {Score},
    %xtick = data,
    ymin=0,
    xmin=0,
    xmax=15,
    legend cell align=left,
    legend style={
      cells={anchor=east},
      legend pos=outer north east
    }
  ]

  \addplot+ [style={rred,mark size=1.5}] table [col sep=semicolon,y=Ground Truth 0,x=Min Limit] {Diagrams/Richard/testMinLimitBC.csv};
  \addplot+ [style={bblue,mark size=1.5}] table [col sep=semicolon,y=Ground Truth Rep 0,x=Min Limit] {Diagrams/Richard/testMinLimitBC.csv};
  \addplot+ [style={ggreen,mark=triangle*,mark size=1.5}] table [col sep=semicolon,y=fMeasure 0,x=Min Limit] {Diagrams/Richard/testMinLimitBC.csv};

  \legend{Precision 0,Recall 0, F-Measure 0}
  
  \end{axis}
\end{tikzpicture}
  \end{center}
  \caption{Performance of the \CTC algorithm for different min limit values for base cluster score with unbounded max limit (max limit = length of longest label).}
  \label{diag:minlimitbcscorerichard}
\end{diagram}

% Max Limit BC Score
\begin{diagram}[H]
  \begin{center}
\begin{tikzpicture}
  \begin{axis}[
    width  = 0.8*\textwidth,
    height = 4.55cm,
    % major x tick style = transparent,
    xlabel = {Max Limit BC Score},
    ylabel = {Score},
    xtick = data,
    ymin=0,
    xmin=3,
    xmax=15,
    legend cell align=left,
    legend style={
      cells={anchor=east},
      legend pos=outer north east
    }
  ]

  \addplot+ [style={rred,mark size=1.5}] table [col sep=semicolon,y=Ground Truth 0,x=Max Limit] {Diagrams/Richard/testMaxLimitBC.csv};
  \addplot+ [style={bblue,mark size=1.5}] table [col sep=semicolon,y=Ground Truth Rep 0,x=Max Limit] {Diagrams/Richard/testMaxLimitBC.csv};
  \addplot+ [style={ggreen,mark=triangle*,mark size=1.5}] table [col sep=semicolon,y=fMeasure 0,x=Max Limit] {Diagrams/Richard/testMaxLimitBC.csv};

  \legend{Precision 0,Recall 0, F-Measure 0}
  
  \end{axis}
\end{tikzpicture}
  \end{center}
  \caption{Performance of the \CTC algorithm for different max limit values for base cluster score. Min limit set to \protect\citeauthor{Oren1998} default.}
  \label{diag:maxlimitbcscorerichard}
\end{diagram}

% Max Limit BC Score best min value
\begin{diagram}[H]
  \begin{center}
\begin{tikzpicture}
  \begin{axis}[
    width  = 0.8*\textwidth,
    height = 4.55cm,
    % major x tick style = transparent,
    xlabel = {Max Limit BC Score},
    ylabel = {Score},
    xtick = data,
    ymin=0,
    xmin=9,
    xmax=15,
    legend cell align=left,
    legend style={
      cells={anchor=east},
      legend pos=outer north east
    }
  ]

  \addplot+ [style={rred,mark size=1.5}] table [col sep=semicolon,y=Ground Truth 0,x=Max Limit] {Diagrams/Richard/testMaxLimitBC2.csv};
  \addplot+ [style={bblue,mark size=1.5}] table [col sep=semicolon,y=Ground Truth Rep 0,x=Max Limit] {Diagrams/Richard/testMaxLimitBC2.csv};
  \addplot+ [style={ggreen,mark=triangle*,mark size=1.5}] table [col sep=semicolon,y=fMeasure 0,x=Max Limit] {Diagrams/Richard/testMaxLimitBC2.csv};

  \legend{Precision 0,Recall 0, F-Measure 0}
  
  \end{axis}
\end{tikzpicture}
  \end{center}
  \caption{Performance of the \CTC algorithm for different max limit values for base cluster score. Min limit set to 8, the best min limit from incremental test on min limit for base cluster score.}
  \label{diag:maxlimitbcscorerichard2}
\end{diagram}

% Drop singleton bc test
\begin{diagram}[H]
  \begin{center}
\begin{tikzpicture}
  \begin{axis}[
    width  = 0.8*\textwidth,
    height = 4.55cm,
    % major x tick style = transparent,
    ybar=2*\pgflinewidth,
    bar width=8pt,
    ymajorgrids = true,
    ylabel = {Score},
    xlabel = {Drop singleton base clusters?},
    symbolic x coords={0,1},
    xtick = data,
    scaled y ticks = false,
    enlarge x limits=0.25,
    ymin=0,
    nodes near coords,
    nodes near coords align={horizontal},
    every node near coord/.append style={font=\tiny,rotate=90,color=black,anchor=west,/pgf/number format/fixed},
    enlarge y limits={upper,value=0.5},
    legend cell align=left,
    legend style={
      cells={anchor=east},
      legend pos=outer north east
    }
  ]

  \addplot [style={rred,fill=rred,mark=none},postaction={pattern=north east lines,pattern color=white}] table [col sep=semicolon,y=Ground Truth 0] {Diagrams/Richard/testDropSingletonBC.csv};
  \addplot [style={bblue,fill=bblue,mark=none},postaction={pattern=north west lines,pattern color=white}] table [col sep=semicolon,y=Ground Truth Rep 0] {Diagrams/Richard/testDropSingletonBC.csv};
  \addplot [style={ggreen,fill=ggreen,mark=none},postaction={pattern=horizontal lines,pattern color=white}] table [col sep=semicolon,y=fMeasure 0] {Diagrams/Richard/testDropSingletonBC.csv};

  \legend{Precision 0,Recall 0, F-Measure 0}
  
  \end{axis}
\end{tikzpicture}
  \end{center}
  \caption{Performance of the \CTC algorithm for exclusion and inclusion of singleton base clusters.}
  %\label{diag:dropsingletonbcrichard}
\end{diagram}

% Drop one word clusters test
\begin{diagram}[H]
  \begin{center}
\begin{tikzpicture}
  \begin{axis}[
    width  = 0.8*\textwidth,
    height = 4.55cm,
    % major x tick style = transparent,
    ybar=2*\pgflinewidth,
    bar width=8pt,
    ymajorgrids = true,
    ylabel = {Score},
    xlabel = {Drop one word clusters?},
    symbolic x coords={0,1},
    xtick = data,
    scaled y ticks = false,
    enlarge x limits=0.25,
    ymin=0,
    nodes near coords,
    nodes near coords align={horizontal},
    every node near coord/.append style={font=\tiny,rotate=90,color=black,anchor=west,/pgf/number format/fixed},
    enlarge y limits={upper,value=0.5},
    legend cell align=left,
    legend style={
      cells={anchor=east},
      legend pos=outer north east
    }
  ]

  \addplot [style={rred,fill=rred,mark=none},postaction={pattern=north east lines,pattern color=white}] table [col sep=semicolon,y=Ground Truth 0] {Diagrams/Richard/testDropOneWordClusters.csv};
  \addplot [style={bblue,fill=bblue,mark=none},postaction={pattern=north west lines,pattern color=white}] table [col sep=semicolon,y=Ground Truth Rep 0] {Diagrams/Richard/testDropOneWordClusters.csv};
  \addplot [style={ggreen,fill=ggreen,mark=none},postaction={pattern=horizontal lines,pattern color=white}] table [col sep=semicolon,y=fMeasure 0] {Diagrams/Richard/testDropOneWordClusters.csv};

  \legend{Precision 0,Recall 0, F-Measure 0}
  
  \end{axis}
\end{tikzpicture}
  \end{center}
  \caption{Performance of the \CTC algorithm on exclusion and inclusion of one word clusters.}
  \label{diag:droponewordclustersrichard}
\end{diagram}

% Sort descending test
\begin{diagram}[H]
  \begin{center}
\begin{tikzpicture}
  \begin{axis}[
    width  = 0.8*\textwidth,
    height = 4.55cm,
    % major x tick style = transparent,
    ybar=2*\pgflinewidth,
    bar width=8pt,
    ymajorgrids = true,
    ylabel = {Score},
    xlabel = {Sort descending?},
    symbolic x coords={0,1},
    xtick = data,
    scaled y ticks = false,
    enlarge x limits=0.25,
    ymin=0,
    nodes near coords,
    nodes near coords align={horizontal},
    every node near coord/.append style={font=\tiny,rotate=90,color=black,anchor=west,/pgf/number format/fixed},
    enlarge y limits={upper,value=0.5},
    legend cell align=left,
    legend style={
      cells={anchor=east},
      legend pos=outer north east
    }
  ]

  \addplot [style={rred,fill=rred,mark=none},postaction={pattern=north east lines,pattern color=white}] table [col sep=semicolon,y=Ground Truth 0] {Diagrams/Richard/testSortDescending.csv};
  \addplot [style={bblue,fill=bblue,mark=none},postaction={pattern=north west lines,pattern color=white}] table [col sep=semicolon,y=Ground Truth Rep 0] {Diagrams/Richard/testSortDescending.csv};
  \addplot [style={ggreen,fill=ggreen,mark=none},postaction={pattern=horizontal lines,pattern color=white}] table [col sep=semicolon,y=fMeasure 0] {Diagrams/Richard/testSortDescending.csv};

  \legend{Precision 0,Recall 0, F-Measure 0}
  
  \end{axis}
\end{tikzpicture}
  \end{center}
  \caption{Performance of the \CTC algorithm when base clusters are sorted in descending and acending order.}
  \label{diag:sortdescendingrichard}
\end{diagram}

% Text amount
\begin{diagram}[H]
  \begin{center}
\begin{tikzpicture}
  \begin{axis}[
    width  = 0.8*\textwidth,
    height = 4.55cm,
    % major x tick style = transparent,
    xlabel = {Article Text Amount},
    xmin=0,
    xmax=1,
    ylabel = {Score},
    %xtick = data,
    ymin=0,
    legend cell align=left,
    legend style={
      cells={anchor=east},
      legend pos=outer north east
    }
  ]

  \addplot+ [style={rred,mark size=1.5}] table [col sep=semicolon,y=Ground Truth 0,x=Article Text Amount] {Diagrams/Richard/testArticleTextAmount.csv};
  \addplot+ [style={bblue,mark size=1.5}] table [col sep=semicolon,y=Ground Truth Rep 0,x=Article Text Amount] {Diagrams/Richard/testArticleTextAmount.csv};
  \addplot+ [style={ggreen,mark=triangle*,mark size=1.5}] table [col sep=semicolon,y=fMeasure 0,x=Article Text Amount] {Diagrams/Richard/testArticleTextAmount.csv};

  \legend{Precision 0,Recall 0, F-Measure 0}
  
  \end{axis}
\end{tikzpicture}
  \end{center}
  \caption{Performance of the \CTC algorithm for different amounts of article text.}
  \label{diag:textamountrichard}
\end{diagram}

% TEXT TYPE TESTS
\begin{diagram}[H]
  \begin{center}
\begin{tikzpicture}
  \begin{axis}[
    width  = 0.8*\textwidth,
    height = 4.55cm,
    % major x tick style = transparent,
    ybar=2*\pgflinewidth,
    bar width=6pt,
    ymajorgrids = true,
    ylabel = {Score},
    symbolic x coords={All,Frontpage,Article sans bread text,Article with bread text,Article text},
    x tick label style={font=\small,text width=1.7cm,align=center},
    xtick = data,
    xlabel = {Text types included},
    scaled y ticks = false,
    enlarge x limits=0.10,
    ymin=0,
    nodes near coords,
    nodes near coords align={horizontal},
    every node near coord/.append style={font=\tiny,rotate=90,color=black,anchor=west,/pgf/number format/fixed},
    enlarge y limits={upper,value=0.5},
    legend cell align=left,
    legend style={
      cells={anchor=east},
      legend pos=outer north east
    }
  ]

  \addplot [style={rred,fill=rred,mark=none},postaction={pattern=north east lines,pattern color=white}] table [col sep=semicolon,y=Ground Truth 0] {Diagrams/Richard/testTextTypes.csv};
  \addplot [style={bblue,fill=bblue,mark=none},postaction={pattern=north west lines,pattern color=white}] table [col sep=semicolon,y=Ground Truth Rep 0] {Diagrams/Richard/testTextTypes.csv};
  \addplot [style={ggreen,fill=ggreen,mark=none},postaction={pattern=horizontal lines,pattern color=white}] table [col sep=semicolon,y=fMeasure 0] {Diagrams/Richard/testTextTypes.csv};

  \legend{Precision 0,Recall 0, F-Measure 0}
  
  \end{axis}
\end{tikzpicture}
  \end{center}
  \caption{Performance of the \CTC algorithm for inclusion of different types of texts.}
  \label{diag:texttypesrichard}
\end{diagram}

% SIMILARITY METHODS TESTS
\begin{diagram}[H]
  \begin{center}
\begin{tikzpicture}
  \begin{axis}[
    width  = 0.8*\textwidth,
    height = 4.55cm,
    % major x tick style = transparent,
    ybar=2*\pgflinewidth,
    bar width=8pt,
    ymajorgrids = true,
    ylabel = {Score},
    xlabel = {Similarity methods},
    symbolic x coords={Etzioni,Jaccard,Cosine,Amendment1C},
    xtick = data,
    scaled y ticks = false,
    enlarge x limits=0.20,
    ymin=0,
    nodes near coords,
    nodes near coords align={horizontal},
    every node near coord/.append style={font=\tiny,rotate=90,color=black,anchor=west, /pgf/number format/fixed},
    enlarge y limits={upper,value=0.5},
    legend cell align=left,
    legend style={
      cells={anchor=east},
      legend pos=outer north east
    }
  ]

  \addplot [style={rred,fill=rred,mark=none},postaction={pattern=north east lines,pattern color=white}] table [col sep=semicolon,y=Ground Truth 0] {Diagrams/Richard/testSimilarityMethods.csv};
  \addplot [style={bblue,fill=bblue,mark=none},postaction={pattern=north west lines,pattern color=white}] table [col sep=semicolon,y=Ground Truth Rep 0] {Diagrams/Richard/testSimilarityMethods.csv};
  \addplot [style={ggreen,fill=ggreen,mark=none},postaction={pattern=horizontal lines,pattern color=white}] table [col sep=semicolon,y=fMeasure 0] {Diagrams/Richard/testSimilarityMethods.csv};

  \legend{Precision 0,Recall 0, F-Measure 0}
  
  \end{axis}
\end{tikzpicture}
  \end{center}
  \caption{Performance of the \CTC algorithm for different similarity methods.}
  %\label{diag:similaritymethodsrichard}
\end{diagram}

% Etzioni THRESHOLD
\begin{diagram}[H]
  \begin{center}
\begin{tikzpicture}
  \begin{axis}[
    % Sizing
    width  = 0.8*\textwidth,
    height = 4.55cm,
    % Data
    xlabel = {Etzioni Similarity Threshold},
    xmin=0,
    xmax=1,
    ymin=0,
    % Labeling
    ylabel = {Score},
    legend cell align=left,
    legend style={
      cells={anchor=east},
      legend pos=outer north east
    }
  ]

  \addplot+ [style={rred,mark size=1.5}] table [col sep=semicolon,y=Ground Truth 0,x=Threshold] {Diagrams/Richard/testEtzioniSimilarity.csv};
  \addplot+ [style={bblue,mark size=1.5}] table [col sep=semicolon,y=Ground Truth Rep 0,x=Threshold] {Diagrams/Richard/testEtzioniSimilarity.csv};
  \addplot+ [style={ggreen,mark=triangle*,mark size=1.5}] table [col sep=semicolon,y=fMeasure 0,x=Threshold] {Diagrams/Richard/testEtzioniSimilarity.csv};

  \legend{Precision 0,Recall 0, F-Measure 0}
  
  \end{axis}
\end{tikzpicture}
  \end{center}
  \caption{Performance of the \CTC algorithm for different Etzioni similarity thresholds.}
  %\label{diag:etzionithresholdrichard}
\end{diagram}


% JACCARD THRESHOLD
\begin{diagram}[H]
  \begin{center}
\begin{tikzpicture}
  \begin{axis}[
    % Sizing
    width  = 0.8*\textwidth,
    height = 4.55cm,
    % Data
    xlabel = {Jaccard Coefficient Threshold},
    xmin=0,
    xmax=1,
    ymin=0,
    % Labeling
    ylabel = {Score},
    legend cell align=left,
    legend style={
      cells={anchor=east},
      legend pos=outer north east
    }
  ]

  \addplot+ [style={rred,mark size=1.5}] table [col sep=semicolon,y=Ground Truth 0,x=Threshold] {Diagrams/Richard/testJaccardSimilarity.csv};
  \addplot+ [style={bblue,mark size=1.5}] table [col sep=semicolon,y=Ground Truth Rep 0,x=Threshold] {Diagrams/Richard/testJaccardSimilarity.csv};
  \addplot+ [style={ggreen,mark=triangle*,mark size=1.5}] table [col sep=semicolon,y=fMeasure 0,x=Threshold] {Diagrams/Richard/testJaccardSimilarity.csv};

  \legend{Precision 0,Recall 0, F-Measure 0}
  
  \end{axis}
\end{tikzpicture}
  \end{center}
  \caption{Performance of the \CTC algorithm for different Jaccard Coefficient thresholds.}
  \label{diag:jaccardthresholdrichard}
\end{diagram}

% COSINE THRESHOLD
\begin{diagram}[H]
  \begin{center}
\begin{tikzpicture}
  \begin{axis}[
    % Sizing
    width  = 0.8*\textwidth,
    height = 4.55cm,
    % Data
    xlabel = {Cosine Threshold},
    xmin=0,
    xmax=1,
    ymin=0,
    % Labeling
    ylabel = {Score},
    legend cell align=left,
    legend style={
      cells={anchor=east},
      legend pos=outer north east
    }
  ]

  \addplot+ [style={rred,mark size=1.5}] table [col sep=semicolon,y=Ground Truth 0,x=Cosine Threshold] {Diagrams/Richard/testCosineSimilarity.csv};
  \addplot+ [style={bblue,mark size=1.5}] table [col sep=semicolon,y=Ground Truth Rep 0,x=Cosine Threshold] {Diagrams/Richard/testCosineSimilarity.csv};
  \addplot+ [style={ggreen,mark=triangle*,mark size=1.5}] table [col sep=semicolon,y=fMeasure 0,x=Cosine Threshold] {Diagrams/Richard/testCosineSimilarity.csv};

  \legend{Precision 0,Recall 0, F-Measure 0}
  
  \end{axis}
\end{tikzpicture}
  \end{center}
  \caption{Performance of the \CTC algorithm for different Cosine Similarity thresholds.}
  \label{diag:cosinethresholdrichard}
\end{diagram}

% Average corpus frequency limit
\begin{diagram}[H]
  \begin{center}
\begin{tikzpicture}
  \begin{axis}[
    width  = 0.8*\textwidth,
    height = 4.55cm,
    % major x tick style = transparent,
    xlabel = {Avg corpus frequency limit},
    ylabel = {Score},
    %xtick = data,
    ymin=0,
    xmin=0,
    xmax=500,
    legend cell align=left,
    legend style={
      cells={anchor=east},
      legend pos=outer north east
    }
  ]

  \addplot+ [style={rred,mark size=1.5}] table [col sep=semicolon,y=Ground Truth 0,x=Avg CF limit] {Diagrams/Richard/testAmendment1CSimilarityAvgCF.csv};
  \addplot+ [style={bblue,mark size=1.5}] table [col sep=semicolon,y=Ground Truth Rep 0,x=Avg CF limit] {Diagrams/Richard/testAmendment1CSimilarityAvgCF.csv};
  \addplot+ [style={ggreen,mark=triangle*,mark size=1.5}] table [col sep=semicolon,y=fMeasure 0,x=Avg CF limit] {Diagrams/Richard/testAmendment1CSimilarityAvgCF.csv};

  \legend{Precision 0,Recall 0, F-Measure 0}
  
  \end{axis}
\end{tikzpicture}
  \end{center}
  \caption{Performance of the \CTC algorithm for different limits on max average corpus frequency in Amendment1C.}
  \label{diag:avgcfamendment1richard}
\end{diagram}

% Base cluster intersect min limit
\begin{diagram}[H]
  \begin{center}
\begin{tikzpicture}
  \begin{axis}[
    width  = 0.8*\textwidth,
    height = 4.55cm,
    % major x tick style = transparent,
    xlabel = {Min label intersect limit},
    ylabel = {Score},
    %xtick = data,
    ymin=0,
    xmin=0,
    xmax=50,
    legend cell align=left,
    legend style={
      cells={anchor=east},
      legend pos=outer north east
    }
  ]

  \addplot+ [style={rred,mark size=1.5}] table [col sep=semicolon,y=Ground Truth 0,x=Min intersect limit] {Diagrams/Richard/testAmendment1CSimilarityIntersect.csv};
  \addplot+ [style={bblue,mark size=1.5}] table [col sep=semicolon,y=Ground Truth Rep 0,x=Min intersect limit] {Diagrams/Richard/testAmendment1CSimilarityIntersect.csv};
  \addplot+ [style={ggreen,mark=triangle*,mark size=1.5}] table [col sep=semicolon,y=fMeasure 0,x=Min intersect limit] {Diagrams/Richard/testAmendment1CSimilarityIntersect.csv};

  \legend{Precision 0,Recall 0, F-Measure 0}
  
  \end{axis}
\end{tikzpicture}
  \end{center}
  \caption{Performance of the \CTC algorithm for different minimum limits on base cluster label intersect in Amendment1C.}
  \label{diag:minintersectamendment1crichard}
\end{diagram}